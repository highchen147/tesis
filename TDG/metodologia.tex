\chapter{METODOLOGÍA}
Para la realización de este trabajo de graduación se necesitarán estudiar artículos y libros que contengan información relacionada a la física computacional, en especial, la rama de la simulación numérica aplicada a ecuaciones de conservación. El artículo \textit{Nonlinear Conservation Laws and Finite Volume Methods} de Randall J. LeVeque \cite{Leveque} inspiró el objetivo general y motivó la realización de este trabajo de graduación, por ende, se tomarán las ideas principales expuestas de dicho artículo para dar estructura a cada uno de los capítulos de este documento. 

Para estudiar la teoría matemática necesaria para justificar los métodos numéricos utilizados es primordial darle un tratamiento especial a la explicación del problema de Riemann. Para que esta sea una explicación adecuada, se le dará una interpretación a través de una ecuación diferencial más simple, esto es, la ecuación de Burgers. Por tanto, las notas de clase \textit{Notes on Burger's Equation} de Maria Cameron \cite{Cameron}, profesora de la universidad de Maryland, serán estudiadas para ejemplificar algunos de los conceptos matemáticos más relevantes de la teoría de las ecuaciones de conservación.

Para conseguir la implementación del método de Roe en la simulación de las Ecuaciones de Euler, se utilizarán dos artículos de Phillip Roe,  \textit{Approximate Riemann Solvers, Parameter Vectors and Difference Schemes} \cite{Roe81} y \textit{Characteristic-Based Schemes for the Euler Equations} \cite{Roe86}. Para escribir el código necesario en \texttt{C++} se hará uso de las notas de clase de Física Computacional, recibidas en 2022.

La librería \textit{PyClaw} servirá para evaluar las simulaciones conseguidas. Se estudiará la librería en el sitio web de la misma con la documentación que se proporciona y el código fuente \cite{clawpack}. 