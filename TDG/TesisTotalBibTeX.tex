%%
%%	AVISO IMPORTANTE
%%	Formato optimizado para el sistema operativo GNU/Linux 64 bits
%%	usar TexLive 2016 (o superior), http://www.ctan.org/tex-archive/systems/texlive/Images/
%%	usar TeXstudio 2.11, http://texstudio.sourceforge.net/

\documentclass[letterpaper,12pt]{thesisECFM}
\usepackage{macros}

\usepackage[spanish, plain, nocomment, noabstract]{flexbib}
\usepackage{physics}
\usepackage{tikz}
\usepackage{graphicx}
\usepackage{siunitx}
\usepackage{subcaption}
\usepackage{hyperref}
\usepackage{listingsutf8}


%%	NO OLVIDE INCLUIR FUENTE DE LAS TABLAS Y FIGURAS

% Decomentar para anular recuadros en los hiperenlaces dentro del pdf
% \hypersetup{pdfborder={0 0 0}}

% Operadores y funciones -------------------------------------------
% Los siguientes son ejemplos de comandos definidos por el usuario
% puede borrarlos, únicamente están para mostrar cómo se construyen con LaTeX
\DeclareMathOperator{\Supp}{Supp}       \DeclareMathOperator{\vol}{Vol}%
\DeclareMathOperator{\Rz}{Re}           \DeclareMathOperator{\Iz}{Im}%

\newcommand{\R}{\mathbb{R}}             \newcommand{\Z}{\mathbb{Z}}%
\newcommand{\C}{\mathbb{C}}             \newcommand{\K}{\mathbb{K}}%
\newcommand{\N}{\mathbb{N}}             \newcommand{\Q}{\mathbb{Q}}%
\newcommand{\Af}{\mathfrak{A}}          \newcommand{\Bf}{\mathfrak{B}}%
\newcommand{\Cf}{\mathfrak{C}}          \newcommand{\Df}{\mathfrak{D}}%
\newcommand{\Ff}{\mathfrak{F}}          \newcommand{\Lf}{\mathfrak{L}}%
\newcommand{\Mf}{\mathfrak{M}}          \newcommand{\Sf}{\mathfrak{S}}%
\newcommand{\Hi}{\mathcal{H}}           \newcommand{\Ba}{\mathcal{B}}%
\newcommand{\nada}{\varnothing}         \newcommand{\To}{\longrightarrow}%
\newcommand{\RR}{[-\infty,+\infty]}     \newcommand{\df}{:=}%
\newcommand{\sani}{$\sigma$\nobreakdash-anillo}
\newcommand{\salg}{$\sigma$\nobreakdash-álgebra}
\newcommand{\ff}{f^{-1}}

%\newcommand{\norm}[1]{\left\Vert#1\right\Vert}
\newcommand{\pnorm}[2]{\norm{#1}_{#2}}
%\newcommand{\abs}[1]{\left\vert#1\right\vert}
\newcommand{\su}[2]{\left\{#1_{#2}\right\}}
\newcommand{\Su}[4]{\suce #1#2 _{#2=#3}^{#4}}
\newcommand{\union}[4]{\bigcup_{{#2}={#3}}^{#4} #1_{#2}}
\newcommand{\Union}[4]{\bigcup \limits_{{#2}={#3}}^{#4} #1_{#2}}
\newcommand{\inter}[4]{\bigcap_{{#2}={#3}}^{#4} #1_{#2}}
\newcommand{\Inter}[4]{\bigcap \limits_{{#2}={#3}}^{#4} #1_{#2}}

\newcommand{\pdual}[2]{\left<#1,#2\right>}
%\newcommand{\bra}[1]{\left<#1\right\vert}
%\newcommand{\ket}[1]{\left\vert#1\right>}
%\newcommand{\braket}[2]{\left<#1\vert#2\right>}
\newcommand{\Braket}[2]{\left\vert#1\right>\! \left<#2\right\vert}
\newcommand\dbar{{\mathchar'26\mkern-9mu \mathrm{d}}}

%%%%%%%%%%%%%%%%%%%%%%%%%%%%%%%%%%%%%%%%%%%%%%%%%%%%%%%%%%%%%%%%%%%%


% Cuerpo de la tesis -----------------------------------------------

\begin{document}

%% Datos generales del trabajo de graduación
\datosThesis%
{1}%						% física 1; matemática 2
{SOLUCIÓN NUMÉRICA DE LAS ECUACIONES DE EULER DE LA DINÁMICA DE FLUIDOS MEDIANTE EL ESQUEMA DE ROE}%		% Título del trabajo de graduación
{RODRIGO RAFAEL CASTILLO CHONG}%			% autor
{DR. ENRIQUE PAZOS ÁVALOS}%		% asesor
{enero de 2024}		% mes y año de la orden de impresión
{2}							% femenino 1; masculino 2

%% Datos generales del examen general privado
\examenPrivado%
{M.Sc. Jorge Marcelo Ixquiac Cabrera}%			% al momento de realizar el examen
{M.Sc. Osmar Obdulio Hernández Aguilar}%		% examinador 1
{Dr. José Rodrigo Sacahui Reyes}%		% examinador 2
{ Dr. Juan Adolfo Ponciano Castellanos}%		% examinador 3
{Ing. Edgar Damián Ochoa Hernández}%			% al momento de realizar el examen
{\onehalfspacing	% interlineado 1 1/2
	
	\OrdenImpresion{ordenImpresion}		% incluye orden de impresión, guardada en pdf
	
	\Agrade{agradecimientos}			% Agradecimientos
	
	\Dedica{dedicatoria}				% Dedicatoria
	
	\par}
 
\frontmatter    % --------------------------------------------------  Hojas preliminares

{\onehalfspacing	% interlineado 1 1/2

\tableofcontents    % Índice general vinculado

%%% \figurasYtablas{ lista_figuras }{ lista_tablas }; con valor 1 se incluye la lista,
%%% cualquier otro valor no la genera
\figurasYtablas{1}{0}

%%% INCLUYA LA SIMBOLOGÍA NECESARIA EN ESTE APARTADO
%%% NO CAMBIAR LA DEFINICIÓN DE LA TABLA LARGA


\chapter{LISTA DE SÍMBOLOS}

\begin{longtable}{@{}l@{\extracolsep{\fill}} p{4.65in} @{}}  %%%	NO CAMBIAR ESTA LÍNEA
  \textsf{Símbolo} & \textsf{Significado}\\[12pt]
  \endhead
  $F_{x}$ & derivada parcial de $F$ respecto a $x$\\
  $\mathbf{U}$ & vector de magnitudes conservadas \\
  ${U}$ & vector de magnitudes conservadas aproximadas numéricamente \\
  $\mathbf{F}$ & vector de flujos de magnitudes conservadas exactas \\
  ${F}$ & vector de flujos de magnitudes conservadas aproximados numéricamente \\
  $\mathbf{A}$ & matriz jacobiana \\
%  $\alpha$ & velocidad de advección \\
  $\varepsilon$ & coeficiente de viscosidad cinemática\\
  $u$ & velocidad del gas sobre el eje $x$\\
  $\rho$ & densidad del gas\\
  $p$ & presión del gas\\
  $e$ & energía interna específica del gas\\
  $E$ & densidad de energía total del gas\\ 
  $T$ & temperatura del gas \\
  $k_B$ & constante de Boltzmann \\
  $\mathcal{R}$ & constante específica de los gases\\
  $\gamma$ & coeficiente de dilatación adiabática
\end{longtable}
  % Lista de símbolos

%%% Haga el diseño que más le guste
\chapter{OBJETIVOS}

\section*{General}
Resolver las ecuaciones de Euler de la dinámica de fluidos para un gas ideal, con el método de volúmenes finitos, utilizando el esquema de Roe.


\section*{Específicos}

\begin{enumerate}
\item Describir el método de volúmenes finitos y la motivación de su uso. 
\item Describir el funcionamiento del esquema de Roe y su implementación en el lenguaje \texttt{C++}.
\item Comparar las soluciones obtenidas a través del programa implementado en \texttt{C++} con las soluciones producidas con la librería PyClaw del lenguaje Python.
\item Analizar la diferencia entre simulaciones considerando gases con distintos grados de libertad, aprovechando la solución numérica obtenida a través del programa escrito en \texttt{C++}.
\end{enumerate}

      % Resumen y objetivos

%%% Haga el diseño que más le guste
\chapter{INTRODUCCIÓN}
El estudio de las ecuaciones diferenciales es de gran importancia en las ciencias físicas, ya que cada teoría física se sustenta en ecuaciones diferenciales que describen el comportamiento a través del tiempo de cualquier sistema que dicha teoría busque explicar. La motivación del estudio de las ecuaciones diferenciales es encontrar soluciones generales de las mismas, principalmente a través de métodos analíticos que buscan soluciones exactas de las ecuaciones diferenciales. Sin embargo no todas las ecuaciones diferenciales poseen soluciones exactas, lo cual motiva el desarrollo de métodos numéricos para la resolución de las mismas.

En el área de estudio del análisis numérico, aplicado a ecuaciones diferenciales, existe una gran variedad de métodos y esquemas de solución, dada la amplia variedad de ecuaciones diferenciales que se hallan en la física. Además, las ecuaciones diferenciales parciales son considerablemente más difíciles de resolver que las ecuaciones diferenciales ordinarias, por lo que existen otros métodos más apropiados para resolver ecuaciones diferenciales que involucran funciones de varias variables.

%. Esto ha conducido al estudio de múltiples métodos analíticos que tienen como objetivo encontrar las soluciones exactas de estas ecuaciones diferenciales. Sin embargo, no todas las ecuaciones diferenciales tienen soluciones analíticas, lo que conlleva a la necesidad de desarrollar métodos para encontrar soluciones numéricas. Es así como nace la rama del análisis numérico aplicado a la resolución de ecuaciones diferenciales.      % Introducción

\mainmatter     % --------------------------------------------------  Cuerpo del Trabajo de Graduación

\chapter{ECUACIONES DE CONSERVACIÓN Y SISTEMAS HIPERBÓLICOS DE PRIMER ORDEN}
En este capítulo se introducen los conceptos fundamentales de las ecuaciones de conservación y sistemas hiperbólicos de primer orden. Se introduce el problema de Riemann asociado a una ecuación de conservación.
\section{Ecuaciones de conservación}
\label{sec:ecuaciones-de-conservacion}
En física, una ecuación de conservación es una ecuación diferencial parcial de la siguiente forma
\begin{equation}
	\pdv{\mathbf{U}}{t} + \pdv{\mathbf{F}(\mathbf{U})}{x} = 0
	\label{eq:conservacion}
\end{equation}
o utilizando una notación más compacta para las derivadas,
\begin{equation}
	\mathbf{U}_{t} + \mathbf{F}(\mathbf{U})_{x} = 0
	\label{eq:conserv-deriv-short}
\end{equation}
donde $\mathbf{U}$ es un vector n-dimensional de variables físicas que se conservan, por ejemplo, la densidad, la masa o el momentum de un medio \cite{Leveque}. En este texto, las variables de las que depende $\mathbf{U}$ dependen de $x$ y $t$, una variable espacial y otra temporal respectivamente. Por tanto, $\mathbf{U}$ se define formalmente como $\mathbf{U} : \mathbb{R} \times  \mathbb{R} \rightarrow \mathbb{R}^{n}$, mientras que la i-ésima variable conservada se denomina $u_{i}$, de tal manera que $\mathbf{U} = \mathbf{U}(u_{1}, u_{2}, \dots, u_{n})$\hspace{2mm}\cite{Leveque}. 

La función $\mathbf{F}$ corresponde al \textbf{flujo} de cada una de las variables involucradas en un punto $(x,t)$ \cite{Leveque}. Al igual que $\mathbf{U}$, la función $\mathbf{F}$ depende de las mismas variables físicas y por ende, también depende de $(x,t)$. Sin embargo, el flujo de cada variable conservada puede tener una forma distinta, entonces es conveniente escribir a $\mathbf{F}$ como un vector de $n$ funciones independientes, $\mathbf{F} = (f_{1}, f_{2}, \dots, f_{n})$
de tal manera que $f_i$ es la función de flujo de la i-ésima variable conservada, $u_i$ \cite{Leveque}.

Una ecuación de conservación para un sistema definido en un intervalo espacial $D = [a,b]$ necesita de condiciones iniciales para su resolución, el caso más simple a considerar es el de un problema de Cauchy \cite{Leveque}. En dicho caso, se debe especificar una función $\mathbf{U}_0(x)$
\begin{equation}
	\mathbf{U}(x,0) = \mathbf{U}_0(x)
\end{equation} 
la cual sea válida para todo $x$ tal que $x \in D$ y condiciones de frontera
\begin{equation}
	\mathbf{U}(a,t) = \mathbf{U}_{a}
\end{equation}
\begin{equation}
	\mathbf{U}(b,t) = \mathbf{U}_{b}
\end{equation}
con $\mathbf{U}_{a}$ y $\mathbf{U}_{b}$ fijos.

Otra forma de escribir una ecuación de conservación es utilizando la matriz jacobiana $\mathbf{A(\mathbf{U})}$ definida como
\begin{equation}
	\mathbf{A(\mathbf{U})} \equiv
	\begin{bmatrix}
		\pdv{f_1}{u_1} & \dots & \pdv{f_1}{u_n} \\
		\vdots & \ddots & \vdots \\
		\pdv{f_n}{u_1} & \dots & \pdv{f_n}{u_n} \\
	\end{bmatrix}
\end{equation}
de tal manera que la ecuación (\ref{eq:conservacion}) se convierte en
\begin{equation}
	\mathbf{U}_{t} + \mathbf{A(\mathbf{U})}\mathbf{U}_{x} = 0
	\label{eq:conservacion-jacobiana}
\end{equation}.
Esta forma de escribir una ecuación de conservación es relevante ya que permite definir un \textbf{sistema hiperbólico}. Un sistema hiperbólico es una ecuación de conservación de la forma (\ref{eq:conservacion-jacobiana}) tal que los autovalores de la matriz $\mathbf{A(\mathbf{U})}$ para todo $\mathbf{U}$ sean reales y que dicha matriz sea diagonalizable \cite{Leveque}. Esto implica que existen $n$ vectores propios linealmente independientes de $\mathbf{A(\mathbf{U})}$ \cite{Leveque}.
Una ecuación de conservación depende de una función $\mathbf{F(\mathbf{U})}$ que, por lo general, no es una función lineal de $\mathbf{U}$, lo que implica que las ecuaciones de conservación son regularmente no lineales \cite{Leveque}. Esto también se puede inferir por la dependencia en $\mathbf{U}$ de la matriz $\mathbf{A}$ en la ecuación (\ref{eq:conservacion-jacobiana}).
\section{Derivación de una ecuación de conservación}
El principio físico de una ecuación de conservación es más explícito cuando esta se deriva a través de cantidades expresadas en forma \textbf{integral.} Considerando un ejemplo de la mecánica de fluidos, se define $M(x_1,x_2,t)$ como la cantidad de masa de un fluido que se encuentra contenido en un ``tubo'' unidimensional en un intervalo  $[x_1,x_2]$ en un tiempo $t$. Si a dicho fluido se le asocia una densidad $\rho(x,t)$, entonces esta última se define de tal manera que su integral definida en un intervalo espacial sea igual a la masa contenida en ese mismo intervalo \cite{Leveque}, i.e.,
\begin{equation}
	M(x_1, x_2, t) = \int_{x_1}^{x_2}\rho(x,t)\dd{x}.
\end{equation}
Ahora, asumiendo que el tubo es cerrado e impenetrable, la cantidad de masa en una región arbitraria $[x_1,x_2]$ puede variar solamente a causa de que el fluido se desplace (fluya) a través de los puntos límites de la región, $x_1$ y $x_2$ \cite{Leveque}. Para cuantificar el flujo que sale o entra en una región se necesita la velocidad del fluido, $v(x,t)$. Cabe destacar que debido a que el fluido se mueve en un espacio unidimensional, su velocidad se limita a dirigirse en el mismo sentido espacial, es decir, su velocidad tiene dirección sobre $x$. Entonces el flujo del fluido en un punto $(x,t)$, $F(x,t)$, se define como 
\begin{equation}
	F(x,t) = \rho(x,t)v(x,t)
\end{equation}\cite{Leveque}.
Entonces, como previamente se comentó, se puede escribir la razón instantánea de cambio de masa en la región $[x_1,x_2]$ en términos del flujo entrante y saliente de la misma región
\begin{equation}
	\dv{t}	\left[M(x_1, x_2, t)\right] = F(x_2,t) - F(x_1,t)
\end{equation}
\begin{equation}
	\dv{t}	\int_{x_1}^{x_2}\rho(x,t)\dd{x} = \rho(x_2,t)v(x_2,t) - \rho(x_1,t)v(x_1,t)
	\label{eq:continuidad-1-integral}
\end{equation}
Esta es la forma integral de una ecuación de conservación \cite{Leveque}. En particular, esta ecuación refleja el principio de conservación de la masa y a su vez es conocida como \textbf{ecuación de continuidad}. La ecuación (\ref{eq:continuidad-1-integral}) se puede integrar en el tiempo para conseguir expresarla independientemente de cualquier derivada, obteniendo
\begin{equation}
	\int_{t_1}^{t_2}\dv{t}\int_{x_1}^{x_2}\rho(x,t)\dd{x}\dd{t}  = \int_{t_1}^{t_2}\left[\rho(x_2,t)v(x_2,t) - \rho(x_1,t)v(x_1,t)\right]\dd{t}
\end{equation}
\begin{equation}
	\int_{x_1}^{x_2}[\rho(x,t_2) - \rho(x,t_1)]\dd{x}  = \int_{t_1}^{t_2}\left[\rho(x_2,t)v(x_2,t) - \rho(x_1,t)v(x_1,t)\right]\dd{t} 
	\label{eq:continuidad-2-integral}.
\end{equation}
Asumiendo que $t_1<t_2$, esta igualdad ofrece una expresión para la diferencia de masa contenida en la región $[x_1,x_2]$ entre los instantes $t_2$ y $t_1$ \cite{Leveque}. 

Es posible obtener una forma diferencial partiendo de la forma integral de una ecuación de conservación, pero es necesario asumir que las funciones $\rho(x,t)$ y $v(x,t)$ son \textbf{diferenciables}. Esta última característica que se exige en las funciones entra en conflicto cuando se estudian soluciones\footnote{Dichas soluciones se conocen como \textbf{soluciones débiles}, tema que se abordará a detalle más adelante.} de las ecuaciones de conservación con discontinuidades. Por lo tanto, la forma integral de las ecuaciones es utilizada al estudiar problemas con dichas características. Para convertir la ecuación en forma integral a su forma diferencial, se tienen que usar las siguientes expresiones:
\begin{equation}
	\rho(x,t_2)-\rho(x,t_1) = \int_{t_1}^{t_2}\pdv{t}\rho(x,t)\dd{t}
\end{equation}
y
\begin{equation}
	\rho(x_2,t)v(x_2,t)-\rho(x_1,t_1)v(x_1,t) = \int_{x_1}^{x_2}\pdv{x}\rho(x,t)v(x,t)\dd{x}
\end{equation}
sustituyendo estas expresiones en (\ref{eq:continuidad-2-integral}) se obtiene
\begin{equation}
	\int_{t_1}^{t_2}\int_{x_1}^{x_2}\left[\pdv{\rho(x,t)}{t} + \pdv{\rho(x,t)v(x,t)}{x}\right]\dd{x}\dd{t} = 0.
	\label{eq:integral-con-derivadas}
\end{equation}
Puesto que la ecuación (\ref{eq:integral-con-derivadas}) se cumple para cualquier punto $(x,t)$ del dominio, el integrando de la misma debe ser idénticamente cero \cite{Leveque}. Entonces,
\begin{equation}
	\rho_{t} + (\rho v)_{x} = 0
\end{equation}
es la forma diferencial de la ecuación de conservación de la masa, así como se definió una ecuación de conservación en (\ref{eq:conserv-deriv-short}), ya que la función de flujo se definió como $F = \rho v$. Puesto que la ecuación diferencial de conservación de la masa involucra dos cantidades físicas, esta se puede resolver ya sea si se conoce previamente la función $v(x,t)$ o si esta última se puede escribir como una función de $\rho$, i.e., $v=v(\rho)$. En este último caso la ecuación de conservación de la masa es una ecuación diferencial parcial únicamente para $\rho$ \cite{Leveque} y toma la siguiente forma
\begin{equation}
	\rho_t + f(\rho)_x = 0.
\end{equation}
Esta última es un ejemplo de \textbf{ecuación de conservación escalar} \cite{Leveque} ya que solamente interviene una incógnita, $\rho$ \cite{Leveque}.
En caso que la velocidad $v(x,t)$ sea una constante $\alpha$, la ecuación de conservación de la masa se convierte en:
\begin{equation}
	\rho_t + \alpha\rho_x = 0,
	\label{eq:adveccion}
\end{equation}
esta ecuación se conoce como \textbf{ecuación de advección lineal} o como ecuación de onda de primer orden \cite{Leveque}\cite{heattransfer}.
\section{Ecuación de advección lineal}
La ecuación de advección (\ref{eq:adveccion}) con la siguiente condición inicial:
\begin{equation}
	\rho(x,0) = \rho_0(x), \hspace{1.7cm} -\infty < x < \infty,
\end{equation}
tiene como solución:
\begin{equation}
	\rho(x,t) = \rho_0(x - \alpha t),
	\label{eq:sol-advec}
\end{equation}
para $t> 0$, asumiendo que $\rho_0$ es una función diferenciable \cite{Leveque}. Esta solución se puede interpretar como la traslación de la función $\rho_0$ a lo largo del eje $x$ con una velocidad $\alpha$, en la misma dirección de la velocidad, es decir, a la derecha si $\alpha > 0$ y a la izquierda si $\alpha < 0$.
\subsection{Curvas características}
Se puede demostrar que la solución (\ref{eq:sol-advec}) es constante respecto al tiempo a lo largo de cada curva definida por $x-\alpha t = x_0$, para cualquier $x_0$ tal que $x_0 \in [-\infty, \infty]$. Dichas curvas de la forma $x=x(t)$ son conocidas como \textbf{curvas características} de la ecuación diferencial en cuestión \cite{Leveque}. Las curvas características definen dominios en donde la ecuación diferencial parcial se puede escribir como una ecuación diferencial ordinaria. Dichas curvas se definen a partir de ecuaciones diferenciales ordinarias. En el caso de la ecuación de advección lineal (\ref{eq:sol-advec}), las características satisfacen la siguiente ecuación diferencial:
\begin{equation}
	\dv{x}{t} = \alpha, \hspace{1.7cm} x(0) = x_0,
\end{equation}
cuya solución es: $x-\alpha t = x_0$ \cite{Leveque}. Para demostrar que la ecuación de advección lineal se convierte en una ecuación diferencial ordinaria a lo largo de dichas curvas, se deriva la función incógnita $\rho$ respecto al tiempo,
\begin{equation}
	\dv{\rho}{t} = \pdv{\rho}{t} + \pdv{\rho}{x}\dv{x}{t}
\end{equation}
\begin{equation}
	\dv{\rho}{t} = \pdv{\rho}{t} + \alpha\pdv{\rho}{x}
\end{equation}
\begin{equation}
	\dv{\rho}{t} = 0,
\end{equation}
este último paso toma efecto dado que a lo largo de las curvas características, $\dv{x}{t} = \alpha$ \cite{Leveque}. Se puede notar que en este último procedimiento se recuperó la ecuación diferencial en cuestión (\ref{eq:sol-advec}), expresada como una ecuación diferencial ordinaria y concluyó al igualar la derivada temporal a cero, consiguiendo demostrar que $\rho$ es constante a lo largo de las características \cite{Leveque}.
\section{Dominio de dependencia de una ecuación de conservación}
En la sección anterior se introdujo el concepto de curvas características de una ecuación de conservación y en el caso de la ecuación de advección, es posible obtener la solución general utilizando la expresión para sus características, esto es $x(t) = x_0 + \alpha t$. Por lo tanto, se puede afirmar que el valor de $\rho(x,t)$ para un instante $(x',t')$ depende de la condición inicial $\rho_0$ únicamente en un solo punto $x'_{0}$ ya que $\rho$ es constante a lo largo de la curva característica $x' = x^{'}_{0} + \alpha t$ \cite{Leveque}. Entonces, sin importar de qué manera se cambien los valores que toma $\rho_0(x)$, mientras no se cambie su valor en $x'_0$, el valor $\rho(x',t')$ permanecerá sin cambiar \cite{Leveque}. 

En el caso general, para cualquier sistema hiperbólico, el valor de la solución $\mathbf{U}$ en un punto $(x',t')$ depende únicamente de los valores que toma la función inicial $\mathbf{U}_0$ en un dominio $D(x',t')$ que corresponde a un intervalo \textbf{cerrado} \cite{Leveque}. Este último se denomina \textbf{dominio de dependencia} \cite{Leveque}. El tamaño de este intervalo usualmente crece a medida que avance el tiempo, sin embargo, el tamaño está acotado de la siguiente forma $D(x',t') = \{x : |x-x'| \leq a_{\text{máx}}t \}$ \cite{Leveque}. En resumen, las ecuaciones hiperbólicas poseen una velocidad finita de propagación; lo que significa que la información de la solución puede viajar a lo largo del espacio a una velocidad $a_{\text{máx}}$ como máximo \cite{Leveque}.

\section{Discontinuidades en soluciones}
Al derivar la forma diferencial de una ecuación de conservación (\ref{eq:integral-con-derivadas}) se enfatizó que era necesario exigir la diferenciabilidad de las cantidades conservadas. Sin embargo, si se toman en cuenta los atributos de la solución, proporcionados por las curvas características, de la ecuación de advección, resulta que el valor de la variable involucrada $\rho(x,t)$ depende únicamente del valor de la condición inicial en un punto $x_0$ tal que $x = x_0 + \alpha t$ \cite{Leveque}. Por esta última razón se puede concluir que, al menos para la ecuación de advección, es posible definir una solución sin exigirle que sea una función  suave \cite{Leveque}. La definición de una solución con singularidades surge gracias a que la forma integral de una ecuación de conservación sí se satisface con funciones no diferenciables, a diferencia de la forma diferencial \cite{Leveque}. También es importante hacer énfasis en que la forma integral de una ecuación de conservación tiene un significado físico más fundamental.

Si la condición inicial de una ecuación de conservación escalar $u_0(x)$ tiene una singularidad (discontinuidad en sí misma o en alguna derivada), en un punto $x_0$, entonces la solución de la ecuación, $u(x,t)$, también poseerá una singularidad del mismo orden en todos los puntos definidos por la característica que parte de $x_0$ \cite{Leveque}. Para los sistemas hiperbólicos en general, las singularidades se propagan a lo largo de las curvas características \cite{Leveque}.

A continuación se describen dos formas de definir las soluciones con singularidades:
\subsection{Viscosidad disipada}
Como se vio anteriormente algunas soluciones de las ecuaciones de conservación pueden poseer singularidades si la condición inicial de la ecuación tiene una singularidad. Para los sistemas hiperbólicos no lineales, las discontinuidades pueden formarse a lo largo de la evolución temporal de la solución \cite{Leveque}. A las singularidades que se desarrollan en las soluciones se les llama \textbf{ondas de choque} o \textbf{shocks} en inglés \cite{Cameron}. Para definir cómo estas funciones siguen siendo soluciones, en el sentido estricto de la palabra, se redefine la ecuación que deben satisfacer agregándole un coeficiente de viscosidad $\varepsilon$ que va acompañado del laplaciano de la variable involucrada \cite{Leveque}. En el caso de la ecuación de advección para $u$, se define la siguiente ecuación modificada
\begin{equation}
	u_t + \alpha u_x = \varepsilon u_{xx},
	\label{eq:advec-difus}
\end{equation}
donde $u_{xx} \equiv \pdv[2]{u}{x}$ \cite{Leveque}. A esta ecuación se le conoce como \textbf{ecuación de advección - difusión}, ya que el término viscoso también se denomina como término difusivo y está involucrado en la ecuación de difusión, también conocida como ecuación de calor \cite{heattransfer}. 

Si se denota a $u^{\varepsilon}(x,t)$ como la solución de la ecuación (\ref{eq:advec-difus}) con condición inicial $u_0(x)$, entonces $u^{\varepsilon} \in C^{\infty}(\mathbb{R} \times [0,\infty)) $ donde $ C^{\infty}(\mathbb{R}\times [0,\infty))$ es el conjunto de funciones infinitamente diferenciables (suaves) definidas en el mismo espacio en donde se resuelve la ecuación \cite{Leveque}. Esto se cumple incluso si $u_0(x)$ no es una función suave, ya que la ecuación (\ref{eq:advec-difus}) es de tipo parabólica \cite{Leveque}. Para encontrar la solución a la ecuación original (\ref{eq:adveccion}) se debe hacer el límite $\varepsilon \rightarrow 0$ en la solución de la ecuación de advección - difusión (\ref{eq:advec-difus}); a esta forma de definir una solución con discontinuidades se le denomina \textbf{solución con viscosidad disipada} \cite{Leveque}.
\subsection{Solución débil}
La otra forma de definir una solución con singularidades es definiendo una \textbf{solución débil} de la ecuación de conservación, haciendo uso de la forma integral de esta última \cite{Leveque}. Es evidente que la forma diferencial de la ecuación de conservación no se satisface con una función con singularidades ya que sus derivadas no están bien definidas en todo el dominio de la función. La idea del procedimiento de la definición de una solución débil es reescribir la ecuación en forma integral, trasladando las derivadas hacia una \textbf{función prueba}, que sí es una función diferenciable y así relajar las condiciones de suavidad para la función incógnita \cite{Leveque}.

Sea $\phi(x,t)$ una función continuamente diferenciable con \textbf{soporte compacto}, i.e., $\phi(x,t) \in C_{0}^{1} (\mathbb{R} \times [0, \infty))$. La tenencia de soporte compacto de la función significa que $\phi(x,t)$ es exactamente igual a cero fuera de algún conjunto cerrado, por lo que el soporte de la función yace en un subconjunto compacto del espacio $(x,t) = \mathbb{R} \times [0, \infty)$ \cite{Cameron}\cite{Leveque}.

Para definir la solución débil de la ecuación de conservación escalar $u_t + f(u)_x = 0$, esta se multiplica por la función $\phi$ descrita anteriormente:
\begin{equation}
	u_t + f(u)_x = 0
	\label{eq:conser-u}
\end{equation}
\begin{equation}
	\phi u_t + \phi f(u)_x = 0.
\end{equation}
Ahora, se integra sobre todo el espacio y el tiempo,
\begin{equation}
	\int_{0}^{\infty}\int_{-\infty}^{\infty}\left(\phi u_t + \phi f(u)_x\right) \dd{x}\dd{t} = 0,
\end{equation}
luego, integrando por partes cada término de la integral, se obtiene
\begin{equation}
	\int_{-\infty}^{\infty}\int_{0}^{\infty} \phi u_t \dd{t}\dd{x} = \int_{-\infty}^{\infty}\int_{0}^{\infty} \left((\phi u)_t - \phi_t u\right) \dd{t}\dd{x}
\end{equation}
\begin{equation}
	\int_{-\infty}^{\infty}\int_{0}^{\infty} \phi u_t \dd{t}\dd{x} = \int_{-\infty}^{\infty} (\phi u) \Big|_{0}^{\infty}\dd{x} - \int_{-\infty}^{\infty}\int_{0}^{\infty}\phi_t u \dd{t}\dd{x}.
\end{equation}
Dado que $\phi$ tiene soporte compacto, es posible asegurarse que $\lim\limits_{t\rightarrow \infty}\phi(x,t) = 0$. Por tanto,
\begin{equation}
	\int_{-\infty}^{\infty}\int_{0}^{\infty} \phi u_t \dd{t}\dd{x} = 
	-\int_{-\infty}^{\infty} \phi(x,0)u(x,0) \dd{x} - \int_{-\infty}^{\infty}\int_{0}^{\infty}\phi_t u \dd{t}\dd{x}.
\end{equation}
Trabajando el segundo término, se tiene
\begin{equation}
	\int_{0}^{\infty}\int_{-\infty}^{\infty}  \phi f(u)_x \dd{x}\dd{t} = 
	\int_{0}^{\infty}\int_{-\infty}^{\infty} \left((\phi f(u))_x - \phi_x f(u)\right) \dd{x}\dd{t}
\end{equation}
\begin{equation}
	\int_{0}^{\infty}\int_{-\infty}^{\infty}  \phi f(u)_x \dd{x}\dd{t} = 
	\int_{0}^{\infty} \phi f(u) \Big|_{-\infty}^{\infty} \dd{t}  - \int_{0}^{\infty}\int_{-\infty}^{\infty} \phi_x f(u) \dd{x}\dd{t},
\end{equation}
de igual manera que con la valuación de la integral del término anterior, el soporte compacto de $\phi$ permite justificar que $\lim\limits_{x\rightarrow -\infty}\phi = 0 = \lim\limits_{x\rightarrow \infty}\phi$. Entonces,
\begin{equation}
	\int_{0}^{\infty}\int_{-\infty}^{\infty}  \phi f(u)_x \dd{x}\dd{t} =  -\int_{0}^{\infty}\int_{-\infty}^{\infty} \phi_x f(u) \dd{x}\dd{t}.
\end{equation}
Por tanto, la integral de la ecuación de conservación multiplicada por la función prueba queda como
\begin{equation}
\begin{aligned}
	\int_{0}^{\infty}\int_{-\infty}^{\infty}\left(\phi u_t + \phi f(u)_x\right) \dd{x}\dd{t} = \int_{-\infty}^{\infty} \phi(x,0)u(x,0) \dd{x} \\
	-\int_{0}^{\infty}\int_{-\infty}^{\infty}\left(\phi_t u + \phi_x f(u)\right) \dd{x}\dd{t} = 0.
\end{aligned}
\end{equation}
Por tanto, se obtiene la igualdad
\begin{equation}
	\int_{0}^{\infty}\int_{-\infty}^{\infty}\left(\phi_t u + \phi_x f(u)\right) \dd{x}\dd{t} = -\int_{-\infty}^{\infty} \phi(x,0)u(x,0) \dd{x},
	\label{eq:weak-equality}
\end{equation}
que tal y como se dijo previamente, las derivadas parciales ahora están siendo aplicadas a $\phi$ y la integral de la nueva ecuación diferencial se iguala a una integral que depende únicamente de las condiciones iniciales del problema a resolver para $u$. Entonces se dice que $u$ es una \textbf{solución débil} de la ecuación (\ref{eq:conser-u}) si y solo si se cumple la igualdad (\ref{eq:weak-equality}) para todo $\phi$ tal que $\phi(x,t) \in C_{0}^{1} (\mathbb{R} \times [0, \infty))$ \cite{Leveque}.

Es posible demostrar que la igualdad (\ref{eq:weak-equality}) es equivalente a la forma integral de la ecuación diferencial en cuestión (ecuación \ref{eq:continuidad-2-integral}), mediante la elección de una función de prueba que esta cumpla con la siguiente propiedad:
\begin{equation}
	\phi(x,t) = 
	\begin{cases}
		1 & \text{para } (x,t) \in [x_1, x_2] \times [t_1, t_2]\\
		0 & \text{para } (x,t) \notin [x_1 - \epsilon, x_2 + \epsilon] \times [t_1 - \epsilon, t_2 + \epsilon].
 	\end{cases}
\end{equation}
Además $\phi$ es una función suave en la banda intermedia con ancho $\epsilon$ \cite{Leveque}. Entonces al calcular la integral (\ref{eq:weak-equality})  se puede notar que esta se reduce a la integral sobre la banda $\epsilon$, a raíz de la propiedad expuesta anteriormente. Al calcular esta integral y haciendo el límite $\epsilon \rightarrow 0$, las derivadas $\phi_x$ y $\phi_t$ tienden a deltas de Dirac \footnote{La delta de Dirac $\delta(x)$ es una función generalizada que cumple con la siguiente propiedad $\int_{-\infty}^{\infty} \delta(x-a) f(x)\dd{x} = f(a)$ \cite{methods}.}, recuperando así los valores de $u$ y $ f(u)$ en las fronteras del dominio $ [x_1, x_2] \times [t_1, t_2]$ en integrales que equivalen a los términos de la ecuación de conservación en forma integral (\ref{eq:continuidad-2-integral})\cite{Leveque}. Cabe destacar que este es un esbozo solamente de la demostración, ya que esta requiere más rigor matemático. Sin embargo, lo imprescindible es destacar la equivalencia entre ambas expresiones.

La solución de viscosidad disipada también califica como una solución débil y físicamente es la solución correcta \cite{Leveque}. Existen muchas soluciones débiles, incluyendo algunas que no tienen un significado físico consistente, para ello es necesario exigir otras condiciones sobre las soluciones débiles de tal manera que se obtenga la solución correcta \cite{Leveque}. Estas condiciones son conocidas como \texttt{condiciones de entropía} por su papel en la dinámica de fluidos.

\section{Ecuación de Burgers}
Se pueden destacar las propiedades más importantes de las ecuaciones de conservación no lineales de la mecánica de fluidos a través de un ejemplo escalar conocido como \textbf{ecuación de Burgers.} Esta ecuación tiene la siguiente forma:

\begin{equation}
	u_t + u u_x = 0,
	\label{eq:burgers}
\end{equation}
la cual es formalmente conocida como Ecuación de Burgers sin viscosidad \cite{Leveque}, pero en este texto se le referirá solamente como Ecuación de Burgers. Esta ecuación es un modelo muy simple de la velocidad $u$ de un gas \cite{Cameron}. La ecuación originalmente estudiada por Jan Burgers \cite{Leveque} contiene un término de viscosidad cinemática:
\begin{equation}
	u_t + u u_x = \varepsilon u_{xx},
	\label{eq:burgers-vis}
\end{equation}
donde $\varepsilon$ es el coeficiente de viscosidad cinemática. A esta última ecuación se le referirá como Ecuación de Burgers viscosa.

De la ecuación (\ref{eq:burgers}) podemos notar un ligero parecido a la ecuación de advección (\ref{eq:adveccion}), con la diferencia que la velocidad de advección $\alpha$ se reemplaza por la variable que sufre la advección, i.e., $u$ \cite{Leveque}, lo que brinda la naturaleza no lineal de este modelo. La ecuación de Burgers se puede escribir en forma conservativa:
\begin{equation}
	u_t + \left(\frac{1}{2}u^{2}\right)_x = 0.
	\label{eq:burgers-cons}
\end{equation}
A partir de esta forma, se puede concluir que la función de flujo es $f(u) = \frac{1}{2}u^{2}$.
\subsection{Curvas características}
Las curvas características de la ecuación de Burgers se pueden definir a través de la siguiente ecuación diferencial \cite{Leveque}
\begin{equation}
	x'(t) = u(x(t),t),
	\label{eq:burgers-char}
\end{equation}
dado que se puede recuperar la ecuación diferencial parcial derivando respecto al tiempo la velocidad:
\begin{equation}
	\dv{t} u(x(t),t) = \pdv{u}{t} + \pdv{u}{x} \dv{x}{t}
\end{equation}
\begin{equation}
	\dv{t} u(x(t),t) = \pdv{u}{t} + u\pdv{u}{x} = 0.
\end{equation}
Por lo tanto, $u$ es constante en el tiempo sobre cualquier característica. Se puede resolver la ecuación (\ref{eq:burgers-char}) utilizando este último hecho:
\begin{equation}
	x(t) = u(\xi,0)t + \xi
	\label{eq:burg-char-expr}
\end{equation}
y al resolver para $\xi$, es posible resolver la ecuación diferencial \cite{Leveque}
\begin{equation}
	u(x,t) = u(\xi, 0).
	\label{eq:sol-burgers-char}
\end{equation}
Es importante destacar que la expresión (\ref{eq:sol-burgers-char}) para $u(x,t)$ no es válida para todo $t$, dado que las características definidas por la ecuación (\ref{eq:burg-char-expr}) pueden intersectarse en algún momento. Esto se da si para algún $x$, se tiene que $u_0'(x) < 0$ \cite{Leveque}. 

\begin{figure}
	\begin{subfigure}{0.5\textwidth}
		\centering
		\includegraphics[width=\linewidth]{../some_plots/cond_inicial.pdf}
		\caption{Gráfica de $u(x,0) = \exp(-x^2)$.}
		\label{fig:subfig1}
	\end{subfigure}
	\begin{subfigure}{0.5\textwidth}
		\centering
		\includegraphics[width=\linewidth]{../some_plots/caracteristicas_burgers.pdf}
		\caption{Características de la ecuación de Burgers.}
		\label{fig:subfig2}
	\end{subfigure}
	\caption{Gráficas de la condición inicial $u(x,0) = \exp(-x^2)$ para la ecuación de Burgers y de algunas características asociadas a esta condición inicial. \textbf{Fuente:} elaboración propia.}
	\label{fig:figura_completa}
\end{figure}
\subsection{Ondas de choque}
Como se puede observar en la figura (\ref{fig:subfig2}), las características asociadas a una condición inicial con forma gaussiana se intersectan. Entonces, si las características se intersectan, la expresión (\ref{eq:sol-burgers-char}) no está bien definida, sino que $u(x,t)$ puede tomar varios valores a la vez, cada uno correspondiente a una característica diferente. En el caso de la ecuación de Burgers, cuando las características se intersectan, la solución se convierte en una ``función'' con tres valores distintos \cite{Leveque}. Esta solución puede tener sentido en algunos contextos, como en el caso del problema de agua superficial \cite{Leveque}. Sin embargo, en la mayoría de casos esta solución no tiene significado físico ya que la variable $u(x,t)$ representa una velocidad y esta debe devolver un solo valor \cite{Leveque}.

La solución con el comportamiento físico correcto cuando las características se intersectan corresponde a una discontinuidad que se propaga con una velocidad determinada \cite{Leveque}. Este efecto se puede demostrar a través del formalismo de la viscosidad disipada. Al obtener la solución de la ecuación de Burgers viscosa (\ref{eq:burgers-vis}), el parámetro $\varepsilon u_{xx}$ evita que la solución presente una discontinuidad, dado que cuando la onda comienza a presentar el choque, la segunda derivada de $u$ crece más rápido que su primera derivada \cite{Leveque}. Así, al efectuar el límite de $\varepsilon \rightarrow 0$ se puede obtener una discontinuidad absoluta que corresponde a la solución de la ecuación de Burgers sin viscosidad. La tendencia de adquirir una discontinuidad  de las soluciones con menos viscosidad se puede evidenciar en la figura (\ref{fig:viscosas}).

\begin{figure}[ht]
	\begin{subfigure}{0.5\textwidth}
		\centering
		\includegraphics[width=\linewidth]{../some_plots/burgers-viscosas/graficas/viscosidades-1.pdf}
		\label{fig:viscosas1}
	\end{subfigure}
	\begin{subfigure}{0.5\textwidth}
		\centering
		\includegraphics[width=\linewidth]{../some_plots/burgers-viscosas/graficas/viscosidades-100.pdf}
		\label{fig:viscosas2}
	\end{subfigure}
	\begin{subfigure}{0.5\textwidth}
		\centering
		\includegraphics[width=\linewidth]{../some_plots/burgers-viscosas/graficas/viscosidades-200.pdf}
		\label{fig:viscosas3}
	\end{subfigure}
	\begin{subfigure}{0.5\textwidth}
		\centering
		\includegraphics[width=\linewidth]{../some_plots/burgers-viscosas/graficas/viscosidades-400.pdf}
		\label{fig:viscosas4}
	\end{subfigure}
	\caption{Gráficas de las soluciones numéricas, en distintos instantes de tiempo, de la ecuación de Burgers viscosa para distintos valores de coeficiente de viscosidad cinemática ($\varepsilon$), incluyendo el caso donde no hay viscosidad (en rojo). \textbf{Fuente:} elaboración propia.}
	\label{fig:viscosas}
\end{figure}

\section{Problema de Riemann}
Como se ha mostrado, las soluciones a las ecuaciones de conservación admiten soluciones débiles que pueden presentar discontinuidades tanto en la condición inicial como en la evolución temporal de las mismas. Por esta razón, es natural estudiar sistemas hiperbólicos con condiciones iniciales discontinuas. Este problema es conocido como \textbf{problema de Riemann}. En el caso escalar, el problema de Riemann para una ecuación de conservación de la forma $u_t + f(u)_x = 0$ corresponde a la siguiente condición inicial
\begin{equation}
	u(x,0) = 
	\begin{cases}
		u_L & \text{si } x < 0 \\
		u_R & \text{si } x \geq 0,
		\label{eq:riemannLR}
	\end{cases}
\end{equation}
en donde la solución es determinada por la relación entre $u_L$ y $u_R$ \cite{Leveque}.
\subsection{Solución al problema de Riemann para la ecuación de Burgers}
Para encontrar la solución al problema de Riemann de la ecuación de Burgers es necesario considerar dos casos.
\subsubsection{Caso $u_L > u_R$:}
La función
\begin{equation}
	u(x,t) = 
	\begin{cases}
		u_L, & \hspace{3mm} x < st \\
		u_R, & \hspace{3mm} x > st,		
	\end{cases}\label{eq:solriemannLR}
\end{equation}
conocida como \textbf{onda de choque}, representa una discontinuidad trasladándose a través del eje $x$ con una velocidad $s$. Esta última es una solución débil de la ecuación de Burgers, para la condición inicial (\ref{eq:riemannLR}), si se cumple la condición general de  \textbf{Rankine - Hugoniot} \cite{Cameron}, que se define como:
\begin{equation}
	f(u_L) - f(u_R) = s(u_L - u_R),
	\label{eq:rankinehugo}
\end{equation}
donde $f(u) = \frac{1}{2} u^{2}$ \cite{Leveque}. A continuación se demuestra la expresión general de la condición Rankine - Hugoniot.\\

\textit{Prueba:} Sea $u$ una solución débil de la ecuación diferencial $u_t + f(u)_x = 0$, de la forma (\ref{eq:solriemannLR}). Sea $\mathcal{M} \gg st$. Aplicando la expresión (\ref{eq:continuidad-1-integral}) a $u$:
\begin{equation}
	\dv{t} \int_{-\mathcal{M}}^{\mathcal{M}}u(x,t)\dd{x} = f(u_L) - f(u_R).
	\label{eq:flux_diff}
\end{equation}
Integrando $u$ a partir de su definición (\ref{eq:solriemannLR}), se tiene
\begin{equation}
	\int_{-\mathcal{M}}^{\mathcal{M}}u(x,t)\dd{x} = (\mathcal{M} + st)u_L + (\mathcal{M} - st)u_R.
\end{equation}
Entonces, sustituyendo en (\ref{eq:flux_diff}),
\begin{equation}
	\dv{t}\left[(\mathcal{M} + st)u_L + (\mathcal{M} - st)u_R\right] = f(u_L) - f(u_R)
\end{equation}
\begin{equation}
	s(u_L - u_R) = f(u_L) - f(u_R),
\end{equation}
se obtiene la condición de salto de Rankine - Hugoniot \cite{Leveque}.

Por lo tanto, para el caso $u_L > u_R$, la solución de la ecuación de Burgers es una discontinuidad que se propaga con una velocidad constante $s = \frac{u_L + u_R}{2}$ \cite{Cameron}.

\begin{figure}[ht]
	\centering
	\includegraphics[width=0.8\linewidth]{../some_plots/cap1/graficas/riemannLR.pdf}
	\caption{Ejemplo gráfico de la solución de la ecuación de Burgers para el caso $u_L > u_R$. \textbf{Fuente: }elaboración propia, basada en el esquema obtenido de \cite{Leveque}.}
	\label{fig:solriemannLR}
\end{figure}

\subsubsection{Caso $u_L < u_R$:}
Hay múltiples soluciones débiles para este caso, por ejemplo, la solución (\ref{eq:solriemannLR}) \cite{Cameron} \cite{Leveque}. Sin embargo, esta no es la solución que corresponde a la solución con viscosidad disipiada \cite{Cameron}. La solución
\begin{equation}
	u(x,t) = 
	\begin{cases}
		u_L, & \hspace{3mm} x < u_{L}t \\
		x/t, & \hspace{3mm} u_{L}t \leq x \leq u_{R}t \\
		u_R, & \hspace{3mm} x > u_{R}t,		
	\end{cases}\label{eq:solriemannRL}
\end{equation}
es la solución correspondiente a la viscosidad disipada y es conocida como una \textbf{onda de rarefacción} \cite{Leveque}. Resulta ser poco práctico tener que recurrir frecuentemente a la solución de viscosidad disipada para encontrar la solución a cualquier caso del problema de Riemann, por lo que en su lugar se suelen imponer condiciones conocidas como \textbf{condiciones de entropía} que son equivalentes a exigir que la solución sea consistente con el formalismo de viscosidad disipada \cite{Leveque}.

\begin{figure}[ht]
	\begin{subfigure}{0.5\textwidth}
		\centering
		\includegraphics[width=\linewidth]{../some_plots/cap1/graficas/riemannRL-0.pdf}
	\end{subfigure}
	\begin{subfigure}{0.5\textwidth}
		\centering
		\includegraphics[width=\linewidth]{../some_plots/cap1/graficas/riemannRL-0.4.pdf}
	\end{subfigure}
	\begin{subfigure}{0.5\textwidth}
		\centering
		\includegraphics[width=\linewidth]{../some_plots/cap1/graficas/riemannRL-0.8.pdf}
	\end{subfigure}
	\begin{subfigure}{0.5\textwidth}
		\centering
		\includegraphics[width=\linewidth]{../some_plots/cap1/graficas/riemannRL-1.pdf}
	\end{subfigure}
	\caption{Gráficas de las soluciones, en distintos instantes de tiempo, de la ecuación de Burgers para el caso $u_L < u_R$ del problema de Riemann \textbf{Fuente:} elaboración propia.}
	\label{fig:rarefaccion}
\end{figure}
\section{Condiciones de entropía}
Se ha mencionado que la solución con viscosidad disipada es la solución débil correcta que debe considerarse al resolver una ecuación de conservación, sin embargo, resulta ser complicado encontrar la solución para una ecuación de conservación con viscosidad. Por lo tanto, se suelen imponer un conjunto de condiciones que sean equivalentes a aplicar el formalismo de viscosidad disipada \cite{Leveque}. Estas se denominan \textbf{condiciones de entropía}, dado que cumplen un papel similar al de la entropía misma en problemas de la dinámica de los gases \cite{Leveque}.

Existen varias versiones de la condición de entropía para una ecuación de conservación. Se mostraran las dos más relevantes con respecto a los métodos numéricos a tratar en este texto.
\subsection{Condición de entropía, primera versión}
La primera versión de la condición de entropía establece que una discontinuidad que se propaga a velocidad $s$ debe cumplir con la siguiente desigualdad
\begin{equation}
	f'(u_L) > s > f'(u_R)
	\label{eq:entropy}
\end{equation}
para que se satisfaga la misma condición \cite{Leveque}. Para interpretar esta condición se debe resaltar que la derivada $f'(u)$ corresponde a la velocidad característica de la onda en un punto $(x,t)$ \cite{Leveque}. Por lo tanto, se puede notar que las curvas características se intersectan con la velocidad de propagación de la discontinuidad, de tal manera que las características a la izquierda de la onda de choque avanzan hacia el mismo choque y las características a la derecha son alcanzadas por la onda de choque \cite{Leveque}.

Por otro lado, si se tuviese una discontinuidad que se propaga con una velocidad $s$ tal que se cumple la condición contraria, i.e., $f'(u_L) < s < f'(u_R)$, esta solución no sería estable ante alguna perturbación, ya sea al añadir una pequeña viscosidad o suavizando la discontinuidad en la solución \cite{Leveque}. Al aplicar dichas perturbaciones, se generarían ondas de rarefacción en lugar de ondas de choque.

\subsection{Condición de entropía, segunda versión}
Esta condición desarrollada por Oleinik \cite{Leveque} indica que si $u(x,t)$ es una solución débil físicamente válida si existe una constante $E >0$  tal que para todo $a>0$, $t>0$, sobre cualquier punto $x$ del dominio espacial se cumple que:
\begin{equation}
	\frac{u(x+a,t) - u(x,t)}{a} < \frac{E}{t}
\end{equation}
\cite{Leveque}.

\section{Sistemas hiperbólicos lineales}
El objetivo de esta sección es explicar la solución general de un sistema hiperbólico lineal y la solución del problema de Riemann asociado. Estos resultados serán de alta utilidad para construir la solución numérica que se busca en este texto.

El sistema descrito en la ecuación (\ref{eq:conserv-deriv-short}) con una función de flujo dada por $\mathbf{F}(\mathbf{U}) = \mathbf{A}\mathbf{U}$, siendo $\mathbf{A}$ una matriz con entradas constantes, es un sistema lineal de conservación. El sistema se puede escribir como
\begin{equation}
	\mathbf{U}_t + \mathbf{A}\mathbf{U}_x = 0
	\label{eq:linear-sys}
\end{equation}
donde $\mathbf{A} \in \mathbb{R}^{m \times m}$ y $\mathbf{U} : \mathbb{R} \times \mathbb{R} \rightarrow \mathbb{R}^{m}$. Formalmente hablando, este es un sistema lineal de $m$ cantidades conservadas \cite{Leveque}. Naturalmente, también se define una condición inicial para este sistema,
\begin{equation}
	\mathbf{U}(x,0) = \mathbf{U}_{0}(x).
	\label{eq:cond-inicial-linear-sys}
\end{equation}

En la sección \ref{sec:ecuaciones-de-conservacion} se definió un sistema hiperbólico de primer orden como un sistema de la forma (\ref{eq:conservacion-jacobiana}) en donde los autovalores de $\mathbf{A}$ son todos distintos y son reales. De tal manera que si el sistema (\ref{eq:linear-sys}) es hiperbólico, se puede diagonalizar $\mathbf{A}$, i.e.,
\begin{equation}
	\mathit{\mathbf{A}} = \mathbf{R}\mathbf{\Lambda} \mathbf{R}^{-1}.
	\label{eq:a-diago}
\end{equation}
En esta relación, $\mathbf{\Lambda}$ es la matriz definida como $\text{diag}(\lambda_1, \lambda_2, \dots \lambda_m)$ donde $\lambda_i$ es el i-ésimo autovalor de $\mathbf{A}$. Mientras que $\mathbf{R}$ es la matriz formada por los autovectores columna de $\mathbf{A}$, denominados $\mathbf{r}_i$ \cite{Leveque}. Por ello, se tiene que
\begin{equation}
	\mathbf{A}\mathbf{r}_{p}  = \lambda_{p} \mathbf{r}_{p} 
\end{equation}
para todo $p \in [1,2,\dots, m]$ \cite{Leveque}.

\subsection{Resolución de un sistema hiperbólico lineal}
Se asumirá que se tiene un sistema de la forma (\ref{eq:linear-sys}), tal que los autovalores de $\mathbf{A}$ son todos distintos y reales, i.e., se trata de un sistema \textbf{estrictamente hiperbólico} \cite{Leveque}. Este sistema se puede resolver introduciendo nuevas variables conocidas como \textbf{variables características} \cite{Leveque}. Se define $\mathbf{v}$ como
\begin{equation}
	\mathbf{v} \equiv \mathbf{R}^{-1} \mathbf{U}.
	\label{eq:def-v}
\end{equation}
Por otro lado, se multiplica la ecuación (\ref{eq:linear-sys}) por $\mathbf{R}$
\begin{equation}
	\mathbf{R}^{-1}\mathbf{U}_t + \mathbf{R}^{-1} \mathbf{A} \mathbf{U}_x = 0,
\end{equation}
sustituyendo la expresión (\ref{eq:a-diago}) obteniendo,
\begin{equation}
	\mathbf{R}^{-1}\mathbf{U}_t + \mathbf{\Lambda} \mathbf{R}^{-1} \mathbf{U}_x = 0
\end{equation}
al usar el hecho de que $\mathbf{R}^{-1} \mathbf{R} = \mathbf{I}$ \cite{Leveque}. Luego, usando la definición de $\mathbf{v}$ (\ref{eq:def-v}) y dado que este ejemplo corresponde a un sistema lineal, las matrices $\mathbf{R}$ y su inversa son constantes, por tanto se tiene que
\begin{equation}
	\mathbf{v}_t + \mathbf{\Lambda} \mathbf{v}_x = 0
	\label{eq:diag-linear-sys}
\end{equation}
\cite{Leveque}. 
Cabe destacar la similitud entre esta última ecuación y la ecuación (\ref{eq:linear-sys}). Sin embargo, la principal diferencia entre éstas es que la matriz $\mathbf{\Lambda}$ es una matriz diagonal y la matriz $\mathbf{A}$, en general, no lo es. Por tanto, el sistema definido en (\ref{eq:diag-linear-sys}) se puede desacoplar, obteniendo $m$ ecuaciones escalares lineales independientes \cite{Leveque}. Esto es,
\begin{equation}
	(v_p)_t + \lambda_{p} (v_{p})_x = 0, \hspace{17mm} \forall p \in [1,2, \dots, m],
\end{equation}
donde $v_p$ es cada componente de $\mathbf{v}$ \cite{Leveque}. Este último sistema se resuelve usando el resultado de la ecuación de advección lineal escalar (\ref{eq:sol-advec}), obteniendo
\begin{equation}
	v_p(x,t) = v_p(x - \lambda_{p}t, 0)
	\label{eq:v-sub-p}
\end{equation}
dado que $v_p(x, 0)$ es la condición inicial para cada ecuación escalar según (\ref{eq:cond-inicial-linear-sys}) \cite{Leveque}.

Ya que se tiene la solución al sistema lineal en términos de las variables características, es oportuno escribir las soluciones en términos de las variables originales, i.e., $\mathbf{U}$. De forma que se tiene,
\begin{equation}
	\mathbf{v}(x,0) = \mathbf{R}^{-1} \mathbf{U}_0(x)
\end{equation}
y
\begin{equation}
	\mathbf{U}(x,t) = \mathbf{R} \mathbf{v}(x,t)
\end{equation}
\cite{Leveque}. Además $\mathbf{U}$ se puede escribir como una combinación lineal de los autovectores de $\mathbf{A}$, tomando en cuenta que el valor $\mathbf{v}_p$ es el coeficiente de cada autovector $\mathbf{r}_p$ en dicha expansión lineal, de tal manera que se tiene lo siguiente
\begin{equation}
	\mathbf{U}(x,t) = \sum_{p=1}^{m} v_p(x,t) \mathbf{r}_p
	\label{eq:sol-linear-sys}
\end{equation}
\begin{equation}
	\mathbf{U}(x,t) = \sum_{p=1}^{m} v_p(x - \lambda_p t, 0) \mathbf{r}_p
	\label{eq:sol-linear-sys-con-caracteristicas}
\end{equation}
\cite{Leveque}. Cabe destacar que, de forma similar al caso escalar, la solución $\mathbf{U}$ a este sistema lineal solo depende de la condición inicial $\mathbf{U}_0$ en los $m$ conjuntos de puntos, dados por $x - \lambda_p t$. Formalmente, su dominio de dependencia se escribe como $D(\bar{x}, \bar{t}) = \{x = \bar{x} - \lambda_{p} \bar{t}, \hspace{3mm} p = 1,2, \dots, m \}$ \cite{Leveque}.

De la misma manera que al generalizar el dominio de dependencia de una ecuación de advección lineal, se pueden introducir las curvas características asociadas a este sistema. El conjunto de características de la $p$ familia está dado por la ecuación $x'(t)=\lambda_{p}$ \cite{Leveque}. De tal manera que dichas curvas son de la forma $x = x_0 + \lambda_{p}t$ y se interpretan como rectas sobre el plano $x-t$, lo cual es consistente con el hecho de que se tratan de las características asociadas a un sistema lineal \cite{Leveque}. El coeficiente $v_p(x,t)$ de cada autovector $\mathbf{r}_p$ es constante a lo largo de cada característica \cite{Leveque}.

\subsection{Solución al problema de Riemann para un sistema hiperbólico lineal}
El problema de Riemann para un sistema hiperbólico lineal corresponde a la ecuación diferencial
\begin{equation}
	\mathbf{U}_t + \mathbf{A}\mathbf{U}_x = 0
\end{equation}
junto a la siguiente condición inicial
\begin{equation}
	\mathbf{U}_{0}(x) = 
	\begin{cases}
		\mathbf{U}_L, & x < 0 \\
		\mathbf{U}_R, & x > 0,
	\end{cases}
\end{equation}
considerando que este sistema es estrictamente hiperbólico \cite{Leveque}.

Para resolver este problema, es conveniente escribir los dos estados $\mathbf{U}_L$ y  $\mathbf{U}_R$ en términos de los autovectores de $\mathbf{A}$, $\mathbf{r}_i$ \cite{Leveque}. Esta expansión se sustenta en la propiedad de independencia lineal que poseen los vectores $\mathbf{r}_i$ y dado que estos son $m$ distintos vectores,  indica que forman una base del espacio $\mathbb{R}^{m}$. Por tanto,
\begin{equation}
	\mathbf{U}_L = \sum_{p=1}^{m} \alpha_{p} \mathbf{r}_{p}
\end{equation}
\begin{equation}
	\mathbf{U}_R = \sum_{p=1}^{m} \beta_{p} \mathbf{r}_{p}
\end{equation}
con $\alpha_p$ y $\beta_{p}$ reales y constantes, para todo $p=1,2,\dots, m$ \cite{Leveque}.
Luego, utilizando la solución (\ref{eq:sol-linear-sys}) se obtiene
\begin{equation}
	v_p(x,0) = 
	\begin{cases}
		\alpha_p, & x < 0 \\
		\beta_{p}, & x > 0,
	\end{cases}
\end{equation}
y
\begin{equation}
	v_p(x,t) = 
	\begin{cases}
		\alpha_p, & \text{si} \hspace{4mm} x - \lambda_{p} t < 0 \\
		\beta_{p}, & \text{si} \hspace{4mm} x - \lambda_{p} t > 0,
	\end{cases}
\end{equation}
utilizando la relación del resultado (\ref{eq:v-sub-p}) \cite{Leveque}. Entonces, la solución al problema de Riemann tiene la forma (\ref{eq:sol-linear-sys-con-caracteristicas}), siendo $v_p$ una función por partes.

Si se define a $P(x,t)$ como el valor máximo de $p$ tal que $x - \lambda_{p} t > 0$, entonces se puede escribir la solución al problema de Riemann como:
\begin{equation}
	\mathbf{U}(x,t) = \sum_{p=1}^{P(x,t)}\beta_{p}\mathbf{r}_{p} + \sum_{p=P(x,t) +1}^{m} \alpha_{p} \mathbf{r}_p
	\label{eq:sol-riemann-linear-sys-1}
\end{equation}
\cite{Leveque}.

Se puede verificar que esta solución cumple con la condición de salto de Rankine-Hugoniot. Notando que al atravesar una curva característica $p$, el coeficiente de $\mathbf{r}_p$ salta de acuerdo a la siguiente expresión:
\begin{equation}
	\mathbf{U}_{Lp} - \mathbf{U}_{Rp} = (\beta_{p} - \alpha_{p})\mathbf{r}_p
\end{equation}
donde $\mathbf{U}_{Lp}$ y $\mathbf{U}_{Rp}$ son los valores de la onda a la izquierda y derecha de la curva característica $p$, respectivamente \cite{Leveque}.

Tomando en cuenta que $\mathbf{F}(\mathbf{U}) = \mathbf{A}\mathbf{U}$, entonces:
\begin{equation}
 	\mathbf{F}(\mathbf{U}_{Lp}) - \mathbf{F}(\mathbf{U}_{Rp}) = \mathbf{A}(\mathbf{U}_{Lp} - \mathbf{U}_{Rp})
\end{equation}
\begin{equation}
	\mathbf{F}(\mathbf{U}_{Lp}) - \mathbf{F}(\mathbf{U}_{Rp}) = \mathbf{A}(\beta_{p} - \alpha_{p})\mathbf{r}_p
\end{equation}
\begin{equation}
	\mathbf{F}(\mathbf{U}_{Lp}) - \mathbf{F}(\mathbf{U}_{Rp}) = \lambda_{p}(\mathbf{U}_{Lp} - \mathbf{U}_{Rp})
\end{equation}
con lo cual se demuestra que se cumple la condición Rankine - Hugoniot, notando que la velocidad de propagación de la discontinuidad es $\lambda_p$ \cite{Leveque}.

Otra forma útil de escribir la solución (\ref{eq:sol-riemann-linear-sys-1}) es a través de los saltos previamente desarrollados, esto es,
\begin{equation}
	\mathbf{U}(x,t) = \mathbf{U}_{L}(x,t) + \sum_{\lambda_{p} < x/t} (\beta_{p} - \alpha_{p}) \mathbf{r}_{p} 
\end{equation}
o, de forma equivalente,
\begin{equation}
	\mathbf{U}(x,t) = \mathbf{U}_{R}(x,t) - \sum_{\lambda_{p} \geq x/t} (\beta_{p} - \alpha_{p}) \mathbf{r}_{p} 
\end{equation}
\cite{Leveque}.      % Cap. 1 

\chapter{MÉTODO DE VOLÚMENES FINITOS Y ESQUEMA DE ROE}
A continuación se describen el método y los esquemas a utilizar para llevar a cabo una solución numérica de una ecuación de conservación. La idea principal del capítulo es describir el método de volúmenes finitos y la motivación de su uso. Se explicarán los esquemas adecuados para aplicar el mencionado método, un solucionador del problema de Riemann, denominado esquema de Godunov y otro solucionador aproximado del problema de Riemann, denominado esquema de Roe. Este último es el esquema elegido para resolver las ecuaciones de Euler en este texto.\\
El contenido de este capítulo se basa en la parte \textit{Numerical Methods} del texto \cite{Leveque} de Randall LeVeque.

\section{Método de volúmenes finitos}
El método de volúmenes finitos (MVF) es un método numérico de integración que se especializa en resolver ecuaciones diferenciales escritas en forma conservativa. El MVF destaca por ofrecer una interpretación peculiar de la función a resolver, ya que es un método basado en la forma \textbf{integral} de las ecuaciones.
\subsection{Discretización del problema}
Al aplicar un método numérico para resolver una ecuación diferencial se necesita discretizar el dominio de la función y la función misma. Considerando a  $D = [a,b]$ como el dominio espacial de la solución de una ecuación de conservación, para aplicar el MVF este dominio se divide en $N$ intervalos iguales llamado \textbf{celdas}. Cada celda se denomina $\mathcal{C}_i$ y se define como un intervalo, $\mathcal{C}_i = [x_i, x_{i+1}]$. Además, cada celta tiene un ancho $h$, dado por
\begin{equation}
	h = \frac{b-a}{N}
\end{equation}
\begin{figure}[ht]
	\centering
	\includegraphics[width=\linewidth]{../some_plots/cap2/graficas/domain.pdf}
	\label{fig:discretizacion-eje-x}
	\caption{Esquema de símbolos usados para la discretización del dominio espacial. \textbf{Fuente:} elaboración propia.}
\end{figure}


      % Cap. 2 

\chapter{ECUACIONES DE EULER Y APLICACIÓN DEL ESQUEMA DE ROE}

En este capítulo se explican y derivan las ecuaciones de Euler utilizando las variables generales (presión, densidad y velocidad) y se introducen las variables conservadas. Se explican las ligaduras adicionales involucradas para que las ecuaciones de Euler sean aplicadas a un gas ideal poliatómico.

Se describe el esquema de Roe implementado en la solución de las ecuaciones de Euler para un gas ideal poliatómico así como las demás especificaciones requeridas por el método de volúmenes finitos. Se explica la implementación del método numérico en \texttt{C++}. Se muestran los resultados obtenidos para un problema de condiciones iniciales específicas.
\section{Ecuaciones de Euler}
Las ecuaciones fundamentales de la dinámica de fluidos se basan en las siguientes leyes de conservación universales:
\begin{itemize}
	\item Conservación de la masa
	\item Conservación del momentum
	\item Conservación de la energía.
\end{itemize}
La ecuación de conservación de la masa, que se derivó en la sección (\ref{sec:derivacion-continuidad}), consiste en aplicar la ecuación de continuidad para un fluido con cierta densidad. La ley de conservación del momentum resulta al aplicar la Segunda Ley de Newton en un fluido. Por último, la ley de conservación de la energía es equivalente a la aplicación de la Primera Ley de la Termodinámica. Además de las tres ecuaciones de conservación, es necesario establecer una relación entre las variables físicas del fluido, de tal manera que el sistema de ecuaciones sea resoluble \cite{heattransfer}. En el caso de un gas ideal, la ecuación adicional que relaciona las variables densidad $\rho$, presión $p$ y temperatura $T$, es la ecuación de estado. 
\subsection{Derivación de las ecuaciones}
%\subsubsection{Ecuación de continuidad}
La ecuación de continuidad, derivada en la sección (\ref{sec:derivacion-continuidad}) tiene la siguiente forma
\begin{equation}
	\rho_{t} + (\rho v)_{x} = 0.
	\label{eq:continuidad-euler}
\end{equation}
Ya que el flujo asociado a esta ecuación de conservación es $\rho v$, se puede interpretar que, generalmente, para cualquier cantidad física $z$ que sufra advección, su flujo estará dado por el producto de la cantidad por la velocidad de advección, i.e., $f=zv$. 

A partir del último razonamiento, el flujo asociado a la advección de momentum $\rho v$, tiene una contribución al flujo dada por $(\rho v) \cdot v = \rho v^2$. Sin embargo, además de la advección que sufre el momentum, deben considerarse las fuerzas que actúan en el fluido para expresar el flujo total del momentum. En este texto no se considerarán fuerzas externas, por lo que únicamente habría que tomar en cuenta la fuerza interna del fluido, que está dada por el gradiente de \textbf{presión}, $p_x$ \cite{LeVeque}. De esta manera, se expresa la ecuación de conservación del momentum:
\begin{equation}
	(\rho v)_t + (\rho v^{2} + p)_{x} = 0.
	\label{eq:momentum-euler}
\end{equation}

Para derivar la conservación de energía, se debe considerar que ésta se compone por un término cinético y uno correspondiente a la energía interna del fluido. Sea $E$ la energía total de un fluido. Entonces, se tiene que:
\begin{equation}
	E = \frac{1}{2}\rho v^{2} + \rho e,
	\label{eq:energia-euler}
\end{equation}
donde el término $\frac{1}{2}\rho v^{2}$ corresponde a la energía cinética por unidad de masa y $e$ es la \textbf{energía interna} por unidad de masa del fluido. Esta última también suele denominarse como energía interna específica. La energía interna depende de los grados de libertad internos de las moléculas de los gases, considerando energía de rotación, cinética, de vibración y otras formas más complejos de energía. Las ecuaciones de Euler asumen que el fluido se encuentra en equilibrio termodinámico y que la ecuación de estado provee una expresión para la energía interna específica tal que ésta depende de la presión y la densidad del fluido únicamente
\begin{equation}
	e = e(p, \rho).
\end{equation}
De igual manera que con el momentum, la energía total se ve afectada por la advección del flujo del fluido. Dado que no se consideran fuerzas externas que afecten al sistema, únicamente la presión del fluido hace trabajo y es proporcional al gradiente de $vp$. Entonces la ecuación de conservación de la energía se reduce a:
\begin{equation}
	E_{t} + [v(E + p)]_{x} = 0.
	\label{eq:energia-euler}
\end{equation}

\subsection{Ecuación de estado para un gas poliatómico}
\section{Aplicación del esquema de Roe}
\subsection{Autovalores y autovectores de $\mathbf{A}(\mathbf{U})$}
\subsection{Valores promediados de Roe}
\subsection{Código implementado}
\section{Resultados}
      % Cap. 3 

\chapter{COMPARACIÓN CON PYCLAW}
Mientras se investigaba sobre las ecuaciones de conservación, previo a la realización de este texto, los textos de Randy LeVeque destacaron por su inmenso aporte a la teoría de la resolución numérica de estos sistemas. Al profundizar en los aportes de LeVeque, fue imposible no toparse con el paquete desarrollado para el lenguaje de programación Python, \textbf{PyClaw}, donde LeVeque está listado como el principal diseñador del software y de los algoritmos implementados \cite{clawpack}. El paquete PyClaw fue desarrollado para la resolución numérica de ecuaciones de conservación lineales y no lineales, utilizando métodos de alta resolución. Estos métodos se basan en la construcción de la solución del problema de Riemann a través de diversos esquemas, como el de Roe.

A continuación se describen los algoritmos implementados en el paquete \textbf{Clawpack}. Esta descripción extrae los conceptos presentados en la documentación oficial del mismo \cite{clawpack}.

\section{Fundamentos de Clawpack}
\textbf{PyClaw} forma parte del paquete de solucionadores numéricos \textbf{Clawpack} \footnote{El nombre abrevia la frase en inglés \textbf{C}onservation \textbf{Law} \textbf{Pack}age.}. Éste es capaz de resolver sistemas de ecuaciones diferenciales de la forma estándar de conservación
\begin{equation}
	q_t + f(q)_x = 0.
\end{equation}

La solución numérica de estos sistemas se basa en solucionadores de Riemann. Sea $S(Q_{i-1}, Q_{i})$ un solucionador de Riemann. Para poder implementar éste último es necesario que retorne un conjunto de $M_{w}$ ondas denominadas $\mathcal{W}_{i-1/2}^{p}$, con velocidades $s_{i-1/2}^{p}$, que correspondan a la solución del problema de Riemann entre las celdas $i$ e $i-1$, siempre que satisfaga la siguiente condición
\begin{equation}
	\sum_{p=1}^{M_{w}}\mathcal{W}_{i-1/2}^{p} = Q_i - Q_{i-1} \equiv \Delta Q_{i-1/2}.
\end{equation}
La construcción, basada en ondas y sus velocidades, de la solución del problema de Riemann se asemeja a la forma en la que se construye la solución para el caso lineal (ver sección \ref{sec:sol-riemann-lineal}). Para calcular la diferencia de flujos que toma lugar en el método de volúmenes finitos, el algoritmo de PyClaw divide la diferencia de flujos en dos fluctuaciones,
\begin{equation}
	\mathcal{A}^{+}\Delta Q_{i-1/2} = \sum_{p}(s_{i-1/2}^{p})^{+}\mathcal{W}_{i-1/2}^{p}
\end{equation}
\begin{equation}
	\mathcal{A}^{-}\Delta Q_{i-1/2} = \sum_{p}(s_{i-1/2}^{p})^{-}\mathcal{W}_{i-1/2}^{p},
\end{equation}
con $s^{-} = \min(s,0)$ y $s^{+} = \max(s,0)$. De tal manera que estas fluctuaciones satisfacen lo siguiente
\begin{equation}
	\mathcal{A}^{+}\Delta Q_{i-1/2} + \mathcal{A}^{-}\Delta Q_{i-1/2} = f(Q_{i})-f(Q_{i-1}).
\end{equation}

Definiendo estos términos, se obtiene una expresión para el esquema general de integración. En la documentación oficial de PyClaw éste se denomina \textbf{Método de Godunov}, pero no debe confundirse con el esquema de Godunov. Entonces,
\begin{equation}
	Q_{i}^{n+1} = Q_{i}^{n} - \frac{k}{h}\left[\mathcal{A}^{+}\Delta Q_{i-1/2} + \mathcal{A}^{-}\Delta Q_{i-1/2}\right]
	\label{eq:pyclaw-scheme}
\end{equation}
es la expresión utilizada en los algoritmos de integración de PyClaw. Cabe resaltar su similitud con \eqref{eq:metodo-vol-finitos-2}, y además implica que la diferencia de flujos numéricos $F(U_{i-1}^n, U_i^n) - F(U_{i}^n, U_{i+1}^n)$ equivale a la suma de las fluctuaciones $\mathcal{A}^{+}\Delta Q_{i-1/2} + \mathcal{A}^{-}\Delta Q_{i-1/2}$ definidas en los algoritmos de PyClaw.

En el software PyClaw se implementa la forma \eqref{eq:pyclaw-scheme} para resolver el sistema de conservación ya que al construir la solución a través de las ondas $\mathcal{W}_{i-1/2}^{p}$ y sus velocidades $s_{i-1/2}^{p}$ es posible implementar métodos de alta resolución. Los métodos de alta resolución introducen valores limitantes para las ondas, de tal manera que se evitan oscilaciones no-naturales cerca de las discontinuidades o también gradientes muy pronunciados o inexactos. Los métodos de alta resolución son métodos numéricos con \textit{variación total disminuida} o métodos \textbf{TVD} por sus siglas en inglés. La forma general de éstos es
\begin{equation}
	Q_{i}^{n+1} = Q_{i}^{n} - \frac{k}{h}\left[\mathcal{A}^{+}\Delta Q_{i-1/2} + \mathcal{A}^{-}\Delta Q_{i-1/2}\right] - \frac{k}{h}\left(\tilde{F}_{i+1/2} - \tilde{F}_{i-1/2}\right),
\end{equation}
donde la corrección de flujos está dada por:
\begin{equation}
	\tilde{F}_{i-1 / 2}=\frac{1}{2} \sum_{p=1}^{M_w}\left|s_{i-1 / 2}^p\right|\left(1-\frac{\Delta t}{\Delta x}\left|s_{i-1 / 2}^p\right|\right) \tilde{\mathcal{W}}_{i-1 / 2}^p .
\end{equation}
Aquí, $\tilde{\mathcal{W}}_{i-1 / 2}^p$ representa la onda ${\mathcal{W}}_{i-1 / 2}^p$ luego de que se le ha aplicado la limitante, de acuerdo al método de alta resolución.
%\clearpage

\section{Código de PyClaw utilizado}
\section{Comparación del primer conjunto de condiciones iniciales}
\section{Comparación del segundo conjunto de condiciones iniciales}
\section{Comparación del tercer conjunto de condiciones iniciales}

\section{Discusión}
%
%\begin{figure}[ht]
%	\centering
%	\includegraphics[width=1.1\linewidth]{../euler1D/plots_en_TDG/py_sin_claw/py_gauss199/1.pdf}
%	\caption{Gráficas para $t=0.0\unit{\s}$}
%\end{figure}
%C:\Users\Rodrigo\Documents\mundo9\Universo\tesis\euler1D\plots_en_TDG\py_sin_claw\py_gauss199\1.pdf

\chapter{SIMULACIONES CON DISTINTOS COEFICIENTES DE DILATACIÓN ADIABÁTICA}
\label{cap:5}
Se comparan los resultados obtenidos en simulaciones del mismo problema de condición inicial pero con distinto coeficiente de dilatación adiabática $\gamma$, esto con el fin de obtener una intuición física, a través de la simulación, de cómo varía el comportamiento de un gas cuando el número de grados de libertad interno del mismo cambia.
\begin{figure}[ht]
	\includegraphics[width=\linewidth]{../euler1D/experimentos/graficas_sod/1.pdf}
	\caption{Primeros tres instantes de las simulaciones con distintos $\gamma$.}
\end{figure}
\begin{figure}[ht]
	\includegraphics[width=\linewidth]{../euler1D/experimentos/graficas_sod/2.pdf}
	\caption{Últimos tres instantes de las simulaciones con distintos $\gamma$.}
\end{figure}

{\backmatter     %	Capítulos no van numerados --------------------------------------------------  Apartados finales

%%% INCLUYA SUS CONCLUSIONES Y RECOMENDACIONES


\chapter{CONCLUSIONES}
\begin{enumerate}
	\item El método de volúmenes finitos es óptimo para resolver ecuaciones diferenciales de conservación. Sin embargo, debe estudiarse adicionalmente un esquema apto para el cálculo de flujos numéricos aproximados entre las celdas definidas por el método.
	\item El esquema de Roe es un método práctico para resolver numéricamente las ecuaciones de Euler y a través de los promedios de Roe, su implementación es directa. La parte más difícil de completar para implementar el esquema de Roe en la resolución de un sistema de conservación, es encontrar el  promedio adecuado para los autovalores y autovectores de la matriz $\hat{\mathbf{A}}$. Sin embargo, las propiedades que se exigen que cumpla $\hat{\mathbf{A}}$ son sencillas de satisfacer y se pueden usar los promedios de Roe para las ecuaciones de Euler como base.
	\item Las diferencias entre las simulaciones de C++ y PyClaw son menores. Sin embargo, la solución de PyClaw se realiza utilizando técnicas de alta resolución; específicamente, produciendo soluciones a segundo orden de precisión. La principal desventaja al utilizar PyClaw es su ineficiente captura de ondas de rarefacción transónica, dado que la librería no posee algoritmos de corrección de entropía implementados.
	\item El uso de las simulaciones de gases poliatómicos como experimentos, a través de la solución numérica de las ecuaciones de Euler, resulta ser una tarea que brinda intuición física suficiente para entender mejor los fenómenos de los medios continuos. A través de las simulaciones con distintos coeficientes de dilatación adiabática, se pudieron observar efectos que parecían alcanzables únicamente por medio de la solución analítica de los problemas. Además, las variables más abstractas, como la entropía, son más fáciles de comprender mediante una gráfica de la misma, obtenida a través de una simulación.
	\item Las simulaciones numéricas no reemplazan completamente los experimentos de laboratorio. Sin embargo, las simulaciones pueden ser de alta utilidad al hacer comparaciones de gases ficticios, tomando como ejemplo las simulaciones de gases con distintos $\gamma$. Las soluciones numéricas son un recurso de alta utilidad didáctica, especialmente en cursos básicos de la carrera de física, como termodinámica o mecánica.
\end{enumerate}

\chapter{RECOMENDACIONES}
\begin{enumerate}
	\item Se recomienda estudiar a mayor profundidad el método de división de flujos, implementado en PyClaw, con la finalidad de añadir algún algoritmo que realice la corrección de entropía. De preferencia, se propone agregar un conjunto de algoritmos correctores de entropía a disposición del solucionador de Euler. Es posible realizar esta adición, ya que la librería Clawpack es un software de código abierto.
	\item Ya que se estudiaron brevemente algunos esquemas distintos al esquema de Roe, se sugiere adaptar el esquema de Godunov para resolver las ecuaciones de Euler. Esto puede realizarse en un programa de C++ o bien, añadiendo un solucionador exacto en el paquete de Clawpack. Sería apropiado analizar y comparar las soluciones producidas con los esquemas de Roe y Godunov.
	\item Es imprescindible considerar la implementación de métodos de impresión de datos más estandarizados, en el software de Clawpack. Esto se expone como una recomendación porque PyClaw resulta ser ligeramente limitante al momento de imprimir los datos de la solución numérica. Es importante reconocer, de todas maneras, que PyClaw se especializa en producir soluciones gráficas a través del paquete de visualización visClaw.
\end{enumerate}
     % Conclusiones y recomendaciones

\renewcommand{\bibname}{BIBLIOGRAF\'IA}
\nocite{*}			% incluir a todas las entredas bibliográficas
%\bibliographystyle{apacite}	% estilo de bibliografía
\bibliographystyle{flexbib}
\bibliography{bbtesis}   % Bibliografía

}

% Descomentar en el caso de necesitar incluir apéndices
%\appendix			% Apéndices

%\chapter{ECUACIONES DE CONSERVACIÓN Y SISTEMAS HIPERBÓLICOS DE PRIMER ORDEN}
En este capítulo se introducen los conceptos fundamentales de las ecuaciones de conservación y sistemas hiperbólicos de primer orden. Se introduce el problema de Riemann asociado a una ecuación de conservación.
\section{Ecuaciones de conservación}
\label{sec:ecuaciones-de-conservacion}
En física, una ecuación de conservación es una ecuación diferencial parcial de la siguiente forma
\begin{equation}
	\pdv{\mathbf{U}}{t} + \pdv{\mathbf{F}(\mathbf{U})}{x} = 0
	\label{eq:conservacion}
\end{equation}
o utilizando una notación más compacta para las derivadas,
\begin{equation}
	\mathbf{U}_{t} + \mathbf{F}(\mathbf{U})_{x} = 0
	\label{eq:conserv-deriv-short}
\end{equation}
donde $\mathbf{U}$ es un vector n-dimensional de variables físicas que se conservan, por ejemplo, la densidad, la masa o el momentum de un medio \cite{Leveque}. En este texto, las variables de las que depende $\mathbf{U}$ dependen de $x$ y $t$, una variable espacial y otra temporal respectivamente. Por tanto, $\mathbf{U}$ se define formalmente como $\mathbf{U} : \mathbb{R} \times  \mathbb{R} \rightarrow \mathbb{R}^{n}$, mientras que la i-ésima variable conservada se denomina $u_{i}$, de tal manera que $\mathbf{U} = \mathbf{U}(u_{1}, u_{2}, \dots, u_{n})$\hspace{2mm}\cite{Leveque}. 

La función $\mathbf{F}$ corresponde al \textbf{flujo} de cada una de las variables involucradas en un punto $(x,t)$ \cite{Leveque}. Al igual que $\mathbf{U}$, la función $\mathbf{F}$ depende de las mismas variables físicas y por ende, también depende de $(x,t)$. Sin embargo, el flujo de cada variable conservada puede tener una forma distinta, entonces es conveniente escribir a $\mathbf{F}$ como un vector de $n$ funciones independientes, $\mathbf{F} = (f_{1}, f_{2}, \dots, f_{n})$
de tal manera que $f_i$ es la función de flujo de la i-ésima variable conservada, $u_i$ \cite{Leveque}.

Una ecuación de conservación para un sistema definido en un intervalo espacial $D = [a,b]$ necesita de condiciones iniciales para su resolución, el caso más simple a considerar es el de un problema de Cauchy \cite{Leveque}. En dicho caso, se debe especificar una función $\mathbf{U}_0(x)$
\begin{equation}
	\mathbf{U}(x,0) = \mathbf{U}_0(x)
\end{equation} 
la cual sea válida para todo $x$ tal que $x \in D$ y condiciones de frontera
\begin{equation}
	\mathbf{U}(a,t) = \mathbf{U}_{a}
\end{equation}
\begin{equation}
	\mathbf{U}(b,t) = \mathbf{U}_{b}
\end{equation}
con $\mathbf{U}_{a}$ y $\mathbf{U}_{b}$ fijos.

Otra forma de escribir una ecuación de conservación es utilizando la matriz jacobiana $\mathbf{A(\mathbf{U})}$ definida como
\begin{equation}
	\mathbf{A(\mathbf{U})} \equiv
	\begin{bmatrix}
		\pdv{f_1}{u_1} & \dots & \pdv{f_1}{u_n} \\
		\vdots & \ddots & \vdots \\
		\pdv{f_n}{u_1} & \dots & \pdv{f_n}{u_n} \\
	\end{bmatrix}
\end{equation}
de tal manera que la ecuación (\ref{eq:conservacion}) se convierte en
\begin{equation}
	\mathbf{U}_{t} + \mathbf{A(\mathbf{U})}\mathbf{U}_{x} = 0
	\label{eq:conservacion-jacobiana}
\end{equation}.
Esta forma de escribir una ecuación de conservación es relevante ya que permite definir un \textbf{sistema hiperbólico}. Un sistema hiperbólico es una ecuación de conservación de la forma (\ref{eq:conservacion-jacobiana}) tal que los autovalores de la matriz $\mathbf{A(\mathbf{U})}$ para todo $\mathbf{U}$ sean reales y que dicha matriz sea diagonalizable \cite{Leveque}. Esto implica que existen $n$ vectores propios linealmente independientes de $\mathbf{A(\mathbf{U})}$ \cite{Leveque}.
Una ecuación de conservación depende de una función $\mathbf{F(\mathbf{U})}$ que, por lo general, no es una función lineal de $\mathbf{U}$, lo que implica que las ecuaciones de conservación son regularmente no lineales \cite{Leveque}. Esto también se puede inferir por la dependencia en $\mathbf{U}$ de la matriz $\mathbf{A}$ en la ecuación (\ref{eq:conservacion-jacobiana}).
\section{Derivación de una ecuación de conservación}
El principio físico de una ecuación de conservación es más explícito cuando esta se deriva a través de cantidades expresadas en forma \textbf{integral.} Considerando un ejemplo de la mecánica de fluidos, se define $M(x_1,x_2,t)$ como la cantidad de masa de un fluido que se encuentra contenido en un ``tubo'' unidimensional en un intervalo  $[x_1,x_2]$ en un tiempo $t$. Si a dicho fluido se le asocia una densidad $\rho(x,t)$, entonces esta última se define de tal manera que su integral definida en un intervalo espacial sea igual a la masa contenida en ese mismo intervalo \cite{Leveque}, i.e.,
\begin{equation}
	M(x_1, x_2, t) = \int_{x_1}^{x_2}\rho(x,t)\dd{x}.
\end{equation}
Ahora, asumiendo que el tubo es cerrado e impenetrable, la cantidad de masa en una región arbitraria $[x_1,x_2]$ puede variar solamente a causa de que el fluido se desplace (fluya) a través de los puntos límites de la región, $x_1$ y $x_2$ \cite{Leveque}. Para cuantificar el flujo que sale o entra en una región se necesita la velocidad del fluido, $v(x,t)$. Cabe destacar que debido a que el fluido se mueve en un espacio unidimensional, su velocidad se limita a dirigirse en el mismo sentido espacial, es decir, su velocidad tiene dirección sobre $x$. Entonces el flujo del fluido en un punto $(x,t)$, $F(x,t)$, se define como 
\begin{equation}
	F(x,t) = \rho(x,t)v(x,t)
\end{equation}\cite{Leveque}.
Entonces, como previamente se comentó, se puede escribir la razón instantánea de cambio de masa en la región $[x_1,x_2]$ en términos del flujo entrante y saliente de la misma región
\begin{equation}
	\dv{t}	\left[M(x_1, x_2, t)\right] = F(x_2,t) - F(x_1,t)
\end{equation}
\begin{equation}
	\dv{t}	\int_{x_1}^{x_2}\rho(x,t)\dd{x} = \rho(x_2,t)v(x_2,t) - \rho(x_1,t)v(x_1,t)
	\label{eq:continuidad-1-integral}
\end{equation}
Esta es la forma integral de una ecuación de conservación \cite{Leveque}. En particular, esta ecuación refleja el principio de conservación de la masa y a su vez es conocida como \textbf{ecuación de continuidad}. La ecuación (\ref{eq:continuidad-1-integral}) se puede integrar en el tiempo para conseguir expresarla independientemente de cualquier derivada, obteniendo
\begin{equation}
	\int_{t_1}^{t_2}\dv{t}\int_{x_1}^{x_2}\rho(x,t)\dd{x}\dd{t}  = \int_{t_1}^{t_2}\left[\rho(x_2,t)v(x_2,t) - \rho(x_1,t)v(x_1,t)\right]\dd{t}
\end{equation}
\begin{equation}
	\int_{x_1}^{x_2}[\rho(x,t_2) - \rho(x,t_1)]\dd{x}  = \int_{t_1}^{t_2}\left[\rho(x_2,t)v(x_2,t) - \rho(x_1,t)v(x_1,t)\right]\dd{t} 
	\label{eq:continuidad-2-integral}.
\end{equation}
Asumiendo que $t_1<t_2$, esta igualdad ofrece una expresión para la diferencia de masa contenida en la región $[x_1,x_2]$ entre los instantes $t_2$ y $t_1$ \cite{Leveque}. 

Es posible obtener una forma diferencial partiendo de la forma integral de una ecuación de conservación, pero es necesario asumir que las funciones $\rho(x,t)$ y $v(x,t)$ son \textbf{diferenciables}. Esta última característica que se exige en las funciones entra en conflicto cuando se estudian soluciones\footnote{Dichas soluciones se conocen como \textbf{soluciones débiles}, tema que se abordará a detalle más adelante.} de las ecuaciones de conservación con discontinuidades. Por lo tanto, la forma integral de las ecuaciones es utilizada al estudiar problemas con dichas características. Para convertir la ecuación en forma integral a su forma diferencial, se tienen que usar las siguientes expresiones:
\begin{equation}
	\rho(x,t_2)-\rho(x,t_1) = \int_{t_1}^{t_2}\pdv{t}\rho(x,t)\dd{t}
\end{equation}
y
\begin{equation}
	\rho(x_2,t)v(x_2,t)-\rho(x_1,t_1)v(x_1,t) = \int_{x_1}^{x_2}\pdv{x}\rho(x,t)v(x,t)\dd{x}
\end{equation}
sustituyendo estas expresiones en (\ref{eq:continuidad-2-integral}) se obtiene
\begin{equation}
	\int_{t_1}^{t_2}\int_{x_1}^{x_2}\left[\pdv{\rho(x,t)}{t} + \pdv{\rho(x,t)v(x,t)}{x}\right]\dd{x}\dd{t} = 0.
	\label{eq:integral-con-derivadas}
\end{equation}
Puesto que la ecuación (\ref{eq:integral-con-derivadas}) se cumple para cualquier punto $(x,t)$ del dominio, el integrando de la misma debe ser idénticamente cero \cite{Leveque}. Entonces,
\begin{equation}
	\rho_{t} + (\rho v)_{x} = 0
\end{equation}
es la forma diferencial de la ecuación de conservación de la masa, así como se definió una ecuación de conservación en (\ref{eq:conserv-deriv-short}), ya que la función de flujo se definió como $F = \rho v$. Puesto que la ecuación diferencial de conservación de la masa involucra dos cantidades físicas, esta se puede resolver ya sea si se conoce previamente la función $v(x,t)$ o si esta última se puede escribir como una función de $\rho$, i.e., $v=v(\rho)$. En este último caso la ecuación de conservación de la masa es una ecuación diferencial parcial únicamente para $\rho$ \cite{Leveque} y toma la siguiente forma
\begin{equation}
	\rho_t + f(\rho)_x = 0.
\end{equation}
Esta última es un ejemplo de \textbf{ecuación de conservación escalar} \cite{Leveque} ya que solamente interviene una incógnita, $\rho$ \cite{Leveque}.
En caso que la velocidad $v(x,t)$ sea una constante $\alpha$, la ecuación de conservación de la masa se convierte en:
\begin{equation}
	\rho_t + \alpha\rho_x = 0,
	\label{eq:adveccion}
\end{equation}
esta ecuación se conoce como \textbf{ecuación de advección lineal} o como ecuación de onda de primer orden \cite{Leveque}\cite{heattransfer}.
\section{Ecuación de advección lineal}
La ecuación de advección (\ref{eq:adveccion}) con la siguiente condición inicial:
\begin{equation}
	\rho(x,0) = \rho_0(x), \hspace{1.7cm} -\infty < x < \infty,
\end{equation}
tiene como solución:
\begin{equation}
	\rho(x,t) = \rho_0(x - \alpha t),
	\label{eq:sol-advec}
\end{equation}
para $t> 0$, asumiendo que $\rho_0$ es una función diferenciable \cite{Leveque}. Esta solución se puede interpretar como la traslación de la función $\rho_0$ a lo largo del eje $x$ con una velocidad $\alpha$, en la misma dirección de la velocidad, es decir, a la derecha si $\alpha > 0$ y a la izquierda si $\alpha < 0$.
\subsection{Curvas características}
Se puede demostrar que la solución (\ref{eq:sol-advec}) es constante respecto al tiempo a lo largo de cada curva definida por $x-\alpha t = x_0$, para cualquier $x_0$ tal que $x_0 \in [-\infty, \infty]$. Dichas curvas de la forma $x=x(t)$ son conocidas como \textbf{curvas características} de la ecuación diferencial en cuestión \cite{Leveque}. Las curvas características definen dominios en donde la ecuación diferencial parcial se puede escribir como una ecuación diferencial ordinaria. Dichas curvas se definen a partir de ecuaciones diferenciales ordinarias. En el caso de la ecuación de advección lineal (\ref{eq:sol-advec}), las características satisfacen la siguiente ecuación diferencial:
\begin{equation}
	\dv{x}{t} = \alpha, \hspace{1.7cm} x(0) = x_0,
\end{equation}
cuya solución es: $x-\alpha t = x_0$ \cite{Leveque}. Para demostrar que la ecuación de advección lineal se convierte en una ecuación diferencial ordinaria a lo largo de dichas curvas, se deriva la función incógnita $\rho$ respecto al tiempo,
\begin{equation}
	\dv{\rho}{t} = \pdv{\rho}{t} + \pdv{\rho}{x}\dv{x}{t}
\end{equation}
\begin{equation}
	\dv{\rho}{t} = \pdv{\rho}{t} + \alpha\pdv{\rho}{x}
\end{equation}
\begin{equation}
	\dv{\rho}{t} = 0,
\end{equation}
este último paso toma efecto dado que a lo largo de las curvas características, $\dv{x}{t} = \alpha$ \cite{Leveque}. Se puede notar que en este último procedimiento se recuperó la ecuación diferencial en cuestión (\ref{eq:sol-advec}), expresada como una ecuación diferencial ordinaria y concluyó al igualar la derivada temporal a cero, consiguiendo demostrar que $\rho$ es constante a lo largo de las características \cite{Leveque}.
\section{Dominio de dependencia de una ecuación de conservación}
En la sección anterior se introdujo el concepto de curvas características de una ecuación de conservación y en el caso de la ecuación de advección, es posible obtener la solución general utilizando la expresión para sus características, esto es $x(t) = x_0 + \alpha t$. Por lo tanto, se puede afirmar que el valor de $\rho(x,t)$ para un instante $(x',t')$ depende de la condición inicial $\rho_0$ únicamente en un solo punto $x'_{0}$ ya que $\rho$ es constante a lo largo de la curva característica $x' = x^{'}_{0} + \alpha t$ \cite{Leveque}. Entonces, sin importar de qué manera se cambien los valores que toma $\rho_0(x)$, mientras no se cambie su valor en $x'_0$, el valor $\rho(x',t')$ permanecerá sin cambiar \cite{Leveque}. 

En el caso general, para cualquier sistema hiperbólico, el valor de la solución $\mathbf{U}$ en un punto $(x',t')$ depende únicamente de los valores que toma la función inicial $\mathbf{U}_0$ en un dominio $D(x',t')$ que corresponde a un intervalo \textbf{cerrado} \cite{Leveque}. Este último se denomina \textbf{dominio de dependencia} \cite{Leveque}. El tamaño de este intervalo usualmente crece a medida que avance el tiempo, sin embargo, el tamaño está acotado de la siguiente forma $D(x',t') = \{x : |x-x'| \leq a_{\text{máx}}t \}$ \cite{Leveque}. En resumen, las ecuaciones hiperbólicas poseen una velocidad finita de propagación; lo que significa que la información de la solución puede viajar a lo largo del espacio a una velocidad $a_{\text{máx}}$ como máximo \cite{Leveque}.

\section{Discontinuidades en soluciones}
Al derivar la forma diferencial de una ecuación de conservación (\ref{eq:integral-con-derivadas}) se enfatizó que era necesario exigir la diferenciabilidad de las cantidades conservadas. Sin embargo, si se toman en cuenta los atributos de la solución, proporcionados por las curvas características, de la ecuación de advección, resulta que el valor de la variable involucrada $\rho(x,t)$ depende únicamente del valor de la condición inicial en un punto $x_0$ tal que $x = x_0 + \alpha t$ \cite{Leveque}. Por esta última razón se puede concluir que, al menos para la ecuación de advección, es posible definir una solución sin exigirle que sea una función  suave \cite{Leveque}. La definición de una solución con singularidades surge gracias a que la forma integral de una ecuación de conservación sí se satisface con funciones no diferenciables, a diferencia de la forma diferencial \cite{Leveque}. También es importante hacer énfasis en que la forma integral de una ecuación de conservación tiene un significado físico más fundamental.

Si la condición inicial de una ecuación de conservación escalar $u_0(x)$ tiene una singularidad (discontinuidad en sí misma o en alguna derivada), en un punto $x_0$, entonces la solución de la ecuación, $u(x,t)$, también poseerá una singularidad del mismo orden en todos los puntos definidos por la característica que parte de $x_0$ \cite{Leveque}. Para los sistemas hiperbólicos en general, las singularidades se propagan a lo largo de las curvas características \cite{Leveque}.

A continuación se describen dos formas de definir las soluciones con singularidades:
\subsection{Viscosidad disipada}
Como se vio anteriormente algunas soluciones de las ecuaciones de conservación pueden poseer singularidades si la condición inicial de la ecuación tiene una singularidad. Para los sistemas hiperbólicos no lineales, las discontinuidades pueden formarse a lo largo de la evolución temporal de la solución \cite{Leveque}. A las singularidades que se desarrollan en las soluciones se les llama \textbf{ondas de choque} o \textbf{shocks} en inglés \cite{Cameron}. Para definir cómo estas funciones siguen siendo soluciones, en el sentido estricto de la palabra, se redefine la ecuación que deben satisfacer agregándole un coeficiente de viscosidad $\varepsilon$ que va acompañado del laplaciano de la variable involucrada \cite{Leveque}. En el caso de la ecuación de advección para $u$, se define la siguiente ecuación modificada
\begin{equation}
	u_t + \alpha u_x = \varepsilon u_{xx},
	\label{eq:advec-difus}
\end{equation}
donde $u_{xx} \equiv \pdv[2]{u}{x}$ \cite{Leveque}. A esta ecuación se le conoce como \textbf{ecuación de advección - difusión}, ya que el término viscoso también se denomina como término difusivo y está involucrado en la ecuación de difusión, también conocida como ecuación de calor \cite{heattransfer}. 

Si se denota a $u^{\varepsilon}(x,t)$ como la solución de la ecuación (\ref{eq:advec-difus}) con condición inicial $u_0(x)$, entonces $u^{\varepsilon} \in C^{\infty}(\mathbb{R} \times [0,\infty)) $ donde $ C^{\infty}(\mathbb{R}\times [0,\infty))$ es el conjunto de funciones infinitamente diferenciables (suaves) definidas en el mismo espacio en donde se resuelve la ecuación \cite{Leveque}. Esto se cumple incluso si $u_0(x)$ no es una función suave, ya que la ecuación (\ref{eq:advec-difus}) es de tipo parabólica \cite{Leveque}. Para encontrar la solución a la ecuación original (\ref{eq:adveccion}) se debe hacer el límite $\varepsilon \rightarrow 0$ en la solución de la ecuación de advección - difusión (\ref{eq:advec-difus}); a esta forma de definir una solución con discontinuidades se le denomina \textbf{solución con viscosidad disipada} \cite{Leveque}.
\subsection{Solución débil}
La otra forma de definir una solución con singularidades es definiendo una \textbf{solución débil} de la ecuación de conservación, haciendo uso de la forma integral de esta última \cite{Leveque}. Es evidente que la forma diferencial de la ecuación de conservación no se satisface con una función con singularidades ya que sus derivadas no están bien definidas en todo el dominio de la función. La idea del procedimiento de la definición de una solución débil es reescribir la ecuación en forma integral, trasladando las derivadas hacia una \textbf{función prueba}, que sí es una función diferenciable y así relajar las condiciones de suavidad para la función incógnita \cite{Leveque}.

Sea $\phi(x,t)$ una función continuamente diferenciable con \textbf{soporte compacto}, i.e., $\phi(x,t) \in C_{0}^{1} (\mathbb{R} \times [0, \infty))$. La tenencia de soporte compacto de la función significa que $\phi(x,t)$ es exactamente igual a cero fuera de algún conjunto cerrado, por lo que el soporte de la función yace en un subconjunto compacto del espacio $(x,t) = \mathbb{R} \times [0, \infty)$ \cite{Cameron}\cite{Leveque}.

Para definir la solución débil de la ecuación de conservación escalar $u_t + f(u)_x = 0$, esta se multiplica por la función $\phi$ descrita anteriormente:
\begin{equation}
	u_t + f(u)_x = 0
	\label{eq:conser-u}
\end{equation}
\begin{equation}
	\phi u_t + \phi f(u)_x = 0.
\end{equation}
Ahora, se integra sobre todo el espacio y el tiempo,
\begin{equation}
	\int_{0}^{\infty}\int_{-\infty}^{\infty}\left(\phi u_t + \phi f(u)_x\right) \dd{x}\dd{t} = 0,
\end{equation}
luego, integrando por partes cada término de la integral, se obtiene
\begin{equation}
	\int_{-\infty}^{\infty}\int_{0}^{\infty} \phi u_t \dd{t}\dd{x} = \int_{-\infty}^{\infty}\int_{0}^{\infty} \left((\phi u)_t - \phi_t u\right) \dd{t}\dd{x}
\end{equation}
\begin{equation}
	\int_{-\infty}^{\infty}\int_{0}^{\infty} \phi u_t \dd{t}\dd{x} = \int_{-\infty}^{\infty} (\phi u) \Big|_{0}^{\infty}\dd{x} - \int_{-\infty}^{\infty}\int_{0}^{\infty}\phi_t u \dd{t}\dd{x}.
\end{equation}
Dado que $\phi$ tiene soporte compacto, es posible asegurarse que $\lim\limits_{t\rightarrow \infty}\phi(x,t) = 0$. Por tanto,
\begin{equation}
	\int_{-\infty}^{\infty}\int_{0}^{\infty} \phi u_t \dd{t}\dd{x} = 
	-\int_{-\infty}^{\infty} \phi(x,0)u(x,0) \dd{x} - \int_{-\infty}^{\infty}\int_{0}^{\infty}\phi_t u \dd{t}\dd{x}.
\end{equation}
Trabajando el segundo término, se tiene
\begin{equation}
	\int_{0}^{\infty}\int_{-\infty}^{\infty}  \phi f(u)_x \dd{x}\dd{t} = 
	\int_{0}^{\infty}\int_{-\infty}^{\infty} \left((\phi f(u))_x - \phi_x f(u)\right) \dd{x}\dd{t}
\end{equation}
\begin{equation}
	\int_{0}^{\infty}\int_{-\infty}^{\infty}  \phi f(u)_x \dd{x}\dd{t} = 
	\int_{0}^{\infty} \phi f(u) \Big|_{-\infty}^{\infty} \dd{t}  - \int_{0}^{\infty}\int_{-\infty}^{\infty} \phi_x f(u) \dd{x}\dd{t},
\end{equation}
de igual manera que con la valuación de la integral del término anterior, el soporte compacto de $\phi$ permite justificar que $\lim\limits_{x\rightarrow -\infty}\phi = 0 = \lim\limits_{x\rightarrow \infty}\phi$. Entonces,
\begin{equation}
	\int_{0}^{\infty}\int_{-\infty}^{\infty}  \phi f(u)_x \dd{x}\dd{t} =  -\int_{0}^{\infty}\int_{-\infty}^{\infty} \phi_x f(u) \dd{x}\dd{t}.
\end{equation}
Por tanto, la integral de la ecuación de conservación multiplicada por la función prueba queda como
\begin{equation}
\begin{aligned}
	\int_{0}^{\infty}\int_{-\infty}^{\infty}\left(\phi u_t + \phi f(u)_x\right) \dd{x}\dd{t} = \int_{-\infty}^{\infty} \phi(x,0)u(x,0) \dd{x} \\
	-\int_{0}^{\infty}\int_{-\infty}^{\infty}\left(\phi_t u + \phi_x f(u)\right) \dd{x}\dd{t} = 0.
\end{aligned}
\end{equation}
Por tanto, se obtiene la igualdad
\begin{equation}
	\int_{0}^{\infty}\int_{-\infty}^{\infty}\left(\phi_t u + \phi_x f(u)\right) \dd{x}\dd{t} = -\int_{-\infty}^{\infty} \phi(x,0)u(x,0) \dd{x},
	\label{eq:weak-equality}
\end{equation}
que tal y como se dijo previamente, las derivadas parciales ahora están siendo aplicadas a $\phi$ y la integral de la nueva ecuación diferencial se iguala a una integral que depende únicamente de las condiciones iniciales del problema a resolver para $u$. Entonces se dice que $u$ es una \textbf{solución débil} de la ecuación (\ref{eq:conser-u}) si y solo si se cumple la igualdad (\ref{eq:weak-equality}) para todo $\phi$ tal que $\phi(x,t) \in C_{0}^{1} (\mathbb{R} \times [0, \infty))$ \cite{Leveque}.

Es posible demostrar que la igualdad (\ref{eq:weak-equality}) es equivalente a la forma integral de la ecuación diferencial en cuestión (ecuación \ref{eq:continuidad-2-integral}), mediante la elección de una función de prueba que esta cumpla con la siguiente propiedad:
\begin{equation}
	\phi(x,t) = 
	\begin{cases}
		1 & \text{para } (x,t) \in [x_1, x_2] \times [t_1, t_2]\\
		0 & \text{para } (x,t) \notin [x_1 - \epsilon, x_2 + \epsilon] \times [t_1 - \epsilon, t_2 + \epsilon].
 	\end{cases}
\end{equation}
Además $\phi$ es una función suave en la banda intermedia con ancho $\epsilon$ \cite{Leveque}. Entonces al calcular la integral (\ref{eq:weak-equality})  se puede notar que esta se reduce a la integral sobre la banda $\epsilon$, a raíz de la propiedad expuesta anteriormente. Al calcular esta integral y haciendo el límite $\epsilon \rightarrow 0$, las derivadas $\phi_x$ y $\phi_t$ tienden a deltas de Dirac \footnote{La delta de Dirac $\delta(x)$ es una función generalizada que cumple con la siguiente propiedad $\int_{-\infty}^{\infty} \delta(x-a) f(x)\dd{x} = f(a)$ \cite{methods}.}, recuperando así los valores de $u$ y $ f(u)$ en las fronteras del dominio $ [x_1, x_2] \times [t_1, t_2]$ en integrales que equivalen a los términos de la ecuación de conservación en forma integral (\ref{eq:continuidad-2-integral})\cite{Leveque}. Cabe destacar que este es un esbozo solamente de la demostración, ya que esta requiere más rigor matemático. Sin embargo, lo imprescindible es destacar la equivalencia entre ambas expresiones.

La solución de viscosidad disipada también califica como una solución débil y físicamente es la solución correcta \cite{Leveque}. Existen muchas soluciones débiles, incluyendo algunas que no tienen un significado físico consistente, para ello es necesario exigir otras condiciones sobre las soluciones débiles de tal manera que se obtenga la solución correcta \cite{Leveque}. Estas condiciones son conocidas como \texttt{condiciones de entropía} por su papel en la dinámica de fluidos.

\section{Ecuación de Burgers}
Se pueden destacar las propiedades más importantes de las ecuaciones de conservación no lineales de la mecánica de fluidos a través de un ejemplo escalar conocido como \textbf{ecuación de Burgers.} Esta ecuación tiene la siguiente forma:

\begin{equation}
	u_t + u u_x = 0,
	\label{eq:burgers}
\end{equation}
la cual es formalmente conocida como Ecuación de Burgers sin viscosidad \cite{Leveque}, pero en este texto se le referirá solamente como Ecuación de Burgers. Esta ecuación es un modelo muy simple de la velocidad $u$ de un gas \cite{Cameron}. La ecuación originalmente estudiada por Jan Burgers \cite{Leveque} contiene un término de viscosidad cinemática:
\begin{equation}
	u_t + u u_x = \varepsilon u_{xx},
	\label{eq:burgers-vis}
\end{equation}
donde $\varepsilon$ es el coeficiente de viscosidad cinemática. A esta última ecuación se le referirá como Ecuación de Burgers viscosa.

De la ecuación (\ref{eq:burgers}) podemos notar un ligero parecido a la ecuación de advección (\ref{eq:adveccion}), con la diferencia que la velocidad de advección $\alpha$ se reemplaza por la variable que sufre la advección, i.e., $u$ \cite{Leveque}, lo que brinda la naturaleza no lineal de este modelo. La ecuación de Burgers se puede escribir en forma conservativa:
\begin{equation}
	u_t + \left(\frac{1}{2}u^{2}\right)_x = 0.
	\label{eq:burgers-cons}
\end{equation}
A partir de esta forma, se puede concluir que la función de flujo es $f(u) = \frac{1}{2}u^{2}$.
\subsection{Curvas características}
Las curvas características de la ecuación de Burgers se pueden definir a través de la siguiente ecuación diferencial \cite{Leveque}
\begin{equation}
	x'(t) = u(x(t),t),
	\label{eq:burgers-char}
\end{equation}
dado que se puede recuperar la ecuación diferencial parcial derivando respecto al tiempo la velocidad:
\begin{equation}
	\dv{t} u(x(t),t) = \pdv{u}{t} + \pdv{u}{x} \dv{x}{t}
\end{equation}
\begin{equation}
	\dv{t} u(x(t),t) = \pdv{u}{t} + u\pdv{u}{x} = 0.
\end{equation}
Por lo tanto, $u$ es constante en el tiempo sobre cualquier característica. Se puede resolver la ecuación (\ref{eq:burgers-char}) utilizando este último hecho:
\begin{equation}
	x(t) = u(\xi,0)t + \xi
	\label{eq:burg-char-expr}
\end{equation}
y al resolver para $\xi$, es posible resolver la ecuación diferencial \cite{Leveque}
\begin{equation}
	u(x,t) = u(\xi, 0).
	\label{eq:sol-burgers-char}
\end{equation}
Es importante destacar que la expresión (\ref{eq:sol-burgers-char}) para $u(x,t)$ no es válida para todo $t$, dado que las características definidas por la ecuación (\ref{eq:burg-char-expr}) pueden intersectarse en algún momento. Esto se da si para algún $x$, se tiene que $u_0'(x) < 0$ \cite{Leveque}. 

\begin{figure}
	\begin{subfigure}{0.5\textwidth}
		\centering
		\includegraphics[width=\linewidth]{../some_plots/cond_inicial.pdf}
		\caption{Gráfica de $u(x,0) = \exp(-x^2)$.}
		\label{fig:subfig1}
	\end{subfigure}
	\begin{subfigure}{0.5\textwidth}
		\centering
		\includegraphics[width=\linewidth]{../some_plots/caracteristicas_burgers.pdf}
		\caption{Características de la ecuación de Burgers.}
		\label{fig:subfig2}
	\end{subfigure}
	\caption{Gráficas de la condición inicial $u(x,0) = \exp(-x^2)$ para la ecuación de Burgers y de algunas características asociadas a esta condición inicial. \textbf{Fuente:} elaboración propia.}
	\label{fig:figura_completa}
\end{figure}
\subsection{Ondas de choque}
Como se puede observar en la figura (\ref{fig:subfig2}), las características asociadas a una condición inicial con forma gaussiana se intersectan. Entonces, si las características se intersectan, la expresión (\ref{eq:sol-burgers-char}) no está bien definida, sino que $u(x,t)$ puede tomar varios valores a la vez, cada uno correspondiente a una característica diferente. En el caso de la ecuación de Burgers, cuando las características se intersectan, la solución se convierte en una ``función'' con tres valores distintos \cite{Leveque}. Esta solución puede tener sentido en algunos contextos, como en el caso del problema de agua superficial \cite{Leveque}. Sin embargo, en la mayoría de casos esta solución no tiene significado físico ya que la variable $u(x,t)$ representa una velocidad y esta debe devolver un solo valor \cite{Leveque}.

La solución con el comportamiento físico correcto cuando las características se intersectan corresponde a una discontinuidad que se propaga con una velocidad determinada \cite{Leveque}. Este efecto se puede demostrar a través del formalismo de la viscosidad disipada. Al obtener la solución de la ecuación de Burgers viscosa (\ref{eq:burgers-vis}), el parámetro $\varepsilon u_{xx}$ evita que la solución presente una discontinuidad, dado que cuando la onda comienza a presentar el choque, la segunda derivada de $u$ crece más rápido que su primera derivada \cite{Leveque}. Así, al efectuar el límite de $\varepsilon \rightarrow 0$ se puede obtener una discontinuidad absoluta que corresponde a la solución de la ecuación de Burgers sin viscosidad. La tendencia de adquirir una discontinuidad  de las soluciones con menos viscosidad se puede evidenciar en la figura (\ref{fig:viscosas}).

\begin{figure}[ht]
	\begin{subfigure}{0.5\textwidth}
		\centering
		\includegraphics[width=\linewidth]{../some_plots/burgers-viscosas/graficas/viscosidades-1.pdf}
		\label{fig:viscosas1}
	\end{subfigure}
	\begin{subfigure}{0.5\textwidth}
		\centering
		\includegraphics[width=\linewidth]{../some_plots/burgers-viscosas/graficas/viscosidades-100.pdf}
		\label{fig:viscosas2}
	\end{subfigure}
	\begin{subfigure}{0.5\textwidth}
		\centering
		\includegraphics[width=\linewidth]{../some_plots/burgers-viscosas/graficas/viscosidades-200.pdf}
		\label{fig:viscosas3}
	\end{subfigure}
	\begin{subfigure}{0.5\textwidth}
		\centering
		\includegraphics[width=\linewidth]{../some_plots/burgers-viscosas/graficas/viscosidades-400.pdf}
		\label{fig:viscosas4}
	\end{subfigure}
	\caption{Gráficas de las soluciones numéricas, en distintos instantes de tiempo, de la ecuación de Burgers viscosa para distintos valores de coeficiente de viscosidad cinemática ($\varepsilon$), incluyendo el caso donde no hay viscosidad (en rojo). \textbf{Fuente:} elaboración propia.}
	\label{fig:viscosas}
\end{figure}

\section{Problema de Riemann}
Como se ha mostrado, las soluciones a las ecuaciones de conservación admiten soluciones débiles que pueden presentar discontinuidades tanto en la condición inicial como en la evolución temporal de las mismas. Por esta razón, es natural estudiar sistemas hiperbólicos con condiciones iniciales discontinuas. Este problema es conocido como \textbf{problema de Riemann}. En el caso escalar, el problema de Riemann para una ecuación de conservación de la forma $u_t + f(u)_x = 0$ corresponde a la siguiente condición inicial
\begin{equation}
	u(x,0) = 
	\begin{cases}
		u_L & \text{si } x < 0 \\
		u_R & \text{si } x \geq 0,
		\label{eq:riemannLR}
	\end{cases}
\end{equation}
en donde la solución es determinada por la relación entre $u_L$ y $u_R$ \cite{Leveque}.
\subsection{Solución al problema de Riemann para la ecuación de Burgers}
Para encontrar la solución al problema de Riemann de la ecuación de Burgers es necesario considerar dos casos.
\subsubsection{Caso $u_L > u_R$:}
La función
\begin{equation}
	u(x,t) = 
	\begin{cases}
		u_L, & \hspace{3mm} x < st \\
		u_R, & \hspace{3mm} x > st,		
	\end{cases}\label{eq:solriemannLR}
\end{equation}
conocida como \textbf{onda de choque}, representa una discontinuidad trasladándose a través del eje $x$ con una velocidad $s$. Esta última es una solución débil de la ecuación de Burgers, para la condición inicial (\ref{eq:riemannLR}), si se cumple la condición general de  \textbf{Rankine - Hugoniot} \cite{Cameron}, que se define como:
\begin{equation}
	f(u_L) - f(u_R) = s(u_L - u_R),
	\label{eq:rankinehugo}
\end{equation}
donde $f(u) = \frac{1}{2} u^{2}$ \cite{Leveque}. A continuación se demuestra la expresión general de la condición Rankine - Hugoniot.\\

\textit{Prueba:} Sea $u$ una solución débil de la ecuación diferencial $u_t + f(u)_x = 0$, de la forma (\ref{eq:solriemannLR}). Sea $\mathcal{M} \gg st$. Aplicando la expresión (\ref{eq:continuidad-1-integral}) a $u$:
\begin{equation}
	\dv{t} \int_{-\mathcal{M}}^{\mathcal{M}}u(x,t)\dd{x} = f(u_L) - f(u_R).
	\label{eq:flux_diff}
\end{equation}
Integrando $u$ a partir de su definición (\ref{eq:solriemannLR}), se tiene
\begin{equation}
	\int_{-\mathcal{M}}^{\mathcal{M}}u(x,t)\dd{x} = (\mathcal{M} + st)u_L + (\mathcal{M} - st)u_R.
\end{equation}
Entonces, sustituyendo en (\ref{eq:flux_diff}),
\begin{equation}
	\dv{t}\left[(\mathcal{M} + st)u_L + (\mathcal{M} - st)u_R\right] = f(u_L) - f(u_R)
\end{equation}
\begin{equation}
	s(u_L - u_R) = f(u_L) - f(u_R),
\end{equation}
se obtiene la condición de salto de Rankine - Hugoniot \cite{Leveque}.

Por lo tanto, para el caso $u_L > u_R$, la solución de la ecuación de Burgers es una discontinuidad que se propaga con una velocidad constante $s = \frac{u_L + u_R}{2}$ \cite{Cameron}.

\begin{figure}[ht]
	\centering
	\includegraphics[width=0.8\linewidth]{../some_plots/cap1/graficas/riemannLR.pdf}
	\caption{Ejemplo gráfico de la solución de la ecuación de Burgers para el caso $u_L > u_R$. \textbf{Fuente: }elaboración propia, basada en el esquema obtenido de \cite{Leveque}.}
	\label{fig:solriemannLR}
\end{figure}

\subsubsection{Caso $u_L < u_R$:}
Hay múltiples soluciones débiles para este caso, por ejemplo, la solución (\ref{eq:solriemannLR}) \cite{Cameron} \cite{Leveque}. Sin embargo, esta no es la solución que corresponde a la solución con viscosidad disipiada \cite{Cameron}. La solución
\begin{equation}
	u(x,t) = 
	\begin{cases}
		u_L, & \hspace{3mm} x < u_{L}t \\
		x/t, & \hspace{3mm} u_{L}t \leq x \leq u_{R}t \\
		u_R, & \hspace{3mm} x > u_{R}t,		
	\end{cases}\label{eq:solriemannRL}
\end{equation}
es la solución correspondiente a la viscosidad disipada y es conocida como una \textbf{onda de rarefacción} \cite{Leveque}. Resulta ser poco práctico tener que recurrir frecuentemente a la solución de viscosidad disipada para encontrar la solución a cualquier caso del problema de Riemann, por lo que en su lugar se suelen imponer condiciones conocidas como \textbf{condiciones de entropía} que son equivalentes a exigir que la solución sea consistente con el formalismo de viscosidad disipada \cite{Leveque}.

\begin{figure}[ht]
	\begin{subfigure}{0.5\textwidth}
		\centering
		\includegraphics[width=\linewidth]{../some_plots/cap1/graficas/riemannRL-0.pdf}
	\end{subfigure}
	\begin{subfigure}{0.5\textwidth}
		\centering
		\includegraphics[width=\linewidth]{../some_plots/cap1/graficas/riemannRL-0.4.pdf}
	\end{subfigure}
	\begin{subfigure}{0.5\textwidth}
		\centering
		\includegraphics[width=\linewidth]{../some_plots/cap1/graficas/riemannRL-0.8.pdf}
	\end{subfigure}
	\begin{subfigure}{0.5\textwidth}
		\centering
		\includegraphics[width=\linewidth]{../some_plots/cap1/graficas/riemannRL-1.pdf}
	\end{subfigure}
	\caption{Gráficas de las soluciones, en distintos instantes de tiempo, de la ecuación de Burgers para el caso $u_L < u_R$ del problema de Riemann \textbf{Fuente:} elaboración propia.}
	\label{fig:rarefaccion}
\end{figure}
\section{Condiciones de entropía}
Se ha mencionado que la solución con viscosidad disipada es la solución débil correcta que debe considerarse al resolver una ecuación de conservación, sin embargo, resulta ser complicado encontrar la solución para una ecuación de conservación con viscosidad. Por lo tanto, se suelen imponer un conjunto de condiciones que sean equivalentes a aplicar el formalismo de viscosidad disipada \cite{Leveque}. Estas se denominan \textbf{condiciones de entropía}, dado que cumplen un papel similar al de la entropía misma en problemas de la dinámica de los gases \cite{Leveque}.

Existen varias versiones de la condición de entropía para una ecuación de conservación. Se mostraran las dos más relevantes con respecto a los métodos numéricos a tratar en este texto.
\subsection{Condición de entropía, primera versión}
La primera versión de la condición de entropía establece que una discontinuidad que se propaga a velocidad $s$ debe cumplir con la siguiente desigualdad
\begin{equation}
	f'(u_L) > s > f'(u_R)
	\label{eq:entropy}
\end{equation}
para que se satisfaga la misma condición \cite{Leveque}. Para interpretar esta condición se debe resaltar que la derivada $f'(u)$ corresponde a la velocidad característica de la onda en un punto $(x,t)$ \cite{Leveque}. Por lo tanto, se puede notar que las curvas características se intersectan con la velocidad de propagación de la discontinuidad, de tal manera que las características a la izquierda de la onda de choque avanzan hacia el mismo choque y las características a la derecha son alcanzadas por la onda de choque \cite{Leveque}.

Por otro lado, si se tuviese una discontinuidad que se propaga con una velocidad $s$ tal que se cumple la condición contraria, i.e., $f'(u_L) < s < f'(u_R)$, esta solución no sería estable ante alguna perturbación, ya sea al añadir una pequeña viscosidad o suavizando la discontinuidad en la solución \cite{Leveque}. Al aplicar dichas perturbaciones, se generarían ondas de rarefacción en lugar de ondas de choque.

\subsection{Condición de entropía, segunda versión}
Esta condición desarrollada por Oleinik \cite{Leveque} indica que si $u(x,t)$ es una solución débil físicamente válida si existe una constante $E >0$  tal que para todo $a>0$, $t>0$, sobre cualquier punto $x$ del dominio espacial se cumple que:
\begin{equation}
	\frac{u(x+a,t) - u(x,t)}{a} < \frac{E}{t}
\end{equation}
\cite{Leveque}.

\section{Sistemas hiperbólicos lineales}
El objetivo de esta sección es explicar la solución general de un sistema hiperbólico lineal y la solución del problema de Riemann asociado. Estos resultados serán de alta utilidad para construir la solución numérica que se busca en este texto.

El sistema descrito en la ecuación (\ref{eq:conserv-deriv-short}) con una función de flujo dada por $\mathbf{F}(\mathbf{U}) = \mathbf{A}\mathbf{U}$, siendo $\mathbf{A}$ una matriz con entradas constantes, es un sistema lineal de conservación. El sistema se puede escribir como
\begin{equation}
	\mathbf{U}_t + \mathbf{A}\mathbf{U}_x = 0
	\label{eq:linear-sys}
\end{equation}
donde $\mathbf{A} \in \mathbb{R}^{m \times m}$ y $\mathbf{U} : \mathbb{R} \times \mathbb{R} \rightarrow \mathbb{R}^{m}$. Formalmente hablando, este es un sistema lineal de $m$ cantidades conservadas \cite{Leveque}. Naturalmente, también se define una condición inicial para este sistema,
\begin{equation}
	\mathbf{U}(x,0) = \mathbf{U}_{0}(x).
	\label{eq:cond-inicial-linear-sys}
\end{equation}

En la sección \ref{sec:ecuaciones-de-conservacion} se definió un sistema hiperbólico de primer orden como un sistema de la forma (\ref{eq:conservacion-jacobiana}) en donde los autovalores de $\mathbf{A}$ son todos distintos y son reales. De tal manera que si el sistema (\ref{eq:linear-sys}) es hiperbólico, se puede diagonalizar $\mathbf{A}$, i.e.,
\begin{equation}
	\mathit{\mathbf{A}} = \mathbf{R}\mathbf{\Lambda} \mathbf{R}^{-1}.
	\label{eq:a-diago}
\end{equation}
En esta relación, $\mathbf{\Lambda}$ es la matriz definida como $\text{diag}(\lambda_1, \lambda_2, \dots \lambda_m)$ donde $\lambda_i$ es el i-ésimo autovalor de $\mathbf{A}$. Mientras que $\mathbf{R}$ es la matriz formada por los autovectores columna de $\mathbf{A}$, denominados $\mathbf{r}_i$ \cite{Leveque}. Por ello, se tiene que
\begin{equation}
	\mathbf{A}\mathbf{r}_{p}  = \lambda_{p} \mathbf{r}_{p} 
\end{equation}
para todo $p \in [1,2,\dots, m]$ \cite{Leveque}.

\subsection{Resolución de un sistema hiperbólico lineal}
Se asumirá que se tiene un sistema de la forma (\ref{eq:linear-sys}), tal que los autovalores de $\mathbf{A}$ son todos distintos y reales, i.e., se trata de un sistema \textbf{estrictamente hiperbólico} \cite{Leveque}. Este sistema se puede resolver introduciendo nuevas variables conocidas como \textbf{variables características} \cite{Leveque}. Se define $\mathbf{v}$ como
\begin{equation}
	\mathbf{v} \equiv \mathbf{R}^{-1} \mathbf{U}.
	\label{eq:def-v}
\end{equation}
Por otro lado, se multiplica la ecuación (\ref{eq:linear-sys}) por $\mathbf{R}$
\begin{equation}
	\mathbf{R}^{-1}\mathbf{U}_t + \mathbf{R}^{-1} \mathbf{A} \mathbf{U}_x = 0,
\end{equation}
sustituyendo la expresión (\ref{eq:a-diago}) obteniendo,
\begin{equation}
	\mathbf{R}^{-1}\mathbf{U}_t + \mathbf{\Lambda} \mathbf{R}^{-1} \mathbf{U}_x = 0
\end{equation}
al usar el hecho de que $\mathbf{R}^{-1} \mathbf{R} = \mathbf{I}$ \cite{Leveque}. Luego, usando la definición de $\mathbf{v}$ (\ref{eq:def-v}) y dado que este ejemplo corresponde a un sistema lineal, las matrices $\mathbf{R}$ y su inversa son constantes, por tanto se tiene que
\begin{equation}
	\mathbf{v}_t + \mathbf{\Lambda} \mathbf{v}_x = 0
	\label{eq:diag-linear-sys}
\end{equation}
\cite{Leveque}. 
Cabe destacar la similitud entre esta última ecuación y la ecuación (\ref{eq:linear-sys}). Sin embargo, la principal diferencia entre éstas es que la matriz $\mathbf{\Lambda}$ es una matriz diagonal y la matriz $\mathbf{A}$, en general, no lo es. Por tanto, el sistema definido en (\ref{eq:diag-linear-sys}) se puede desacoplar, obteniendo $m$ ecuaciones escalares lineales independientes \cite{Leveque}. Esto es,
\begin{equation}
	(v_p)_t + \lambda_{p} (v_{p})_x = 0, \hspace{17mm} \forall p \in [1,2, \dots, m],
\end{equation}
donde $v_p$ es cada componente de $\mathbf{v}$ \cite{Leveque}. Este último sistema se resuelve usando el resultado de la ecuación de advección lineal escalar (\ref{eq:sol-advec}), obteniendo
\begin{equation}
	v_p(x,t) = v_p(x - \lambda_{p}t, 0)
	\label{eq:v-sub-p}
\end{equation}
dado que $v_p(x, 0)$ es la condición inicial para cada ecuación escalar según (\ref{eq:cond-inicial-linear-sys}) \cite{Leveque}.

Ya que se tiene la solución al sistema lineal en términos de las variables características, es oportuno escribir las soluciones en términos de las variables originales, i.e., $\mathbf{U}$. De forma que se tiene,
\begin{equation}
	\mathbf{v}(x,0) = \mathbf{R}^{-1} \mathbf{U}_0(x)
\end{equation}
y
\begin{equation}
	\mathbf{U}(x,t) = \mathbf{R} \mathbf{v}(x,t)
\end{equation}
\cite{Leveque}. Además $\mathbf{U}$ se puede escribir como una combinación lineal de los autovectores de $\mathbf{A}$, tomando en cuenta que el valor $\mathbf{v}_p$ es el coeficiente de cada autovector $\mathbf{r}_p$ en dicha expansión lineal, de tal manera que se tiene lo siguiente
\begin{equation}
	\mathbf{U}(x,t) = \sum_{p=1}^{m} v_p(x,t) \mathbf{r}_p
	\label{eq:sol-linear-sys}
\end{equation}
\begin{equation}
	\mathbf{U}(x,t) = \sum_{p=1}^{m} v_p(x - \lambda_p t, 0) \mathbf{r}_p
	\label{eq:sol-linear-sys-con-caracteristicas}
\end{equation}
\cite{Leveque}. Cabe destacar que, de forma similar al caso escalar, la solución $\mathbf{U}$ a este sistema lineal solo depende de la condición inicial $\mathbf{U}_0$ en los $m$ conjuntos de puntos, dados por $x - \lambda_p t$. Formalmente, su dominio de dependencia se escribe como $D(\bar{x}, \bar{t}) = \{x = \bar{x} - \lambda_{p} \bar{t}, \hspace{3mm} p = 1,2, \dots, m \}$ \cite{Leveque}.

De la misma manera que al generalizar el dominio de dependencia de una ecuación de advección lineal, se pueden introducir las curvas características asociadas a este sistema. El conjunto de características de la $p$ familia está dado por la ecuación $x'(t)=\lambda_{p}$ \cite{Leveque}. De tal manera que dichas curvas son de la forma $x = x_0 + \lambda_{p}t$ y se interpretan como rectas sobre el plano $x-t$, lo cual es consistente con el hecho de que se tratan de las características asociadas a un sistema lineal \cite{Leveque}. El coeficiente $v_p(x,t)$ de cada autovector $\mathbf{r}_p$ es constante a lo largo de cada característica \cite{Leveque}.

\subsection{Solución al problema de Riemann para un sistema hiperbólico lineal}
El problema de Riemann para un sistema hiperbólico lineal corresponde a la ecuación diferencial
\begin{equation}
	\mathbf{U}_t + \mathbf{A}\mathbf{U}_x = 0
\end{equation}
junto a la siguiente condición inicial
\begin{equation}
	\mathbf{U}_{0}(x) = 
	\begin{cases}
		\mathbf{U}_L, & x < 0 \\
		\mathbf{U}_R, & x > 0,
	\end{cases}
\end{equation}
considerando que este sistema es estrictamente hiperbólico \cite{Leveque}.

Para resolver este problema, es conveniente escribir los dos estados $\mathbf{U}_L$ y  $\mathbf{U}_R$ en términos de los autovectores de $\mathbf{A}$, $\mathbf{r}_i$ \cite{Leveque}. Esta expansión se sustenta en la propiedad de independencia lineal que poseen los vectores $\mathbf{r}_i$ y dado que estos son $m$ distintos vectores,  indica que forman una base del espacio $\mathbb{R}^{m}$. Por tanto,
\begin{equation}
	\mathbf{U}_L = \sum_{p=1}^{m} \alpha_{p} \mathbf{r}_{p}
\end{equation}
\begin{equation}
	\mathbf{U}_R = \sum_{p=1}^{m} \beta_{p} \mathbf{r}_{p}
\end{equation}
con $\alpha_p$ y $\beta_{p}$ reales y constantes, para todo $p=1,2,\dots, m$ \cite{Leveque}.
Luego, utilizando la solución (\ref{eq:sol-linear-sys}) se obtiene
\begin{equation}
	v_p(x,0) = 
	\begin{cases}
		\alpha_p, & x < 0 \\
		\beta_{p}, & x > 0,
	\end{cases}
\end{equation}
y
\begin{equation}
	v_p(x,t) = 
	\begin{cases}
		\alpha_p, & \text{si} \hspace{4mm} x - \lambda_{p} t < 0 \\
		\beta_{p}, & \text{si} \hspace{4mm} x - \lambda_{p} t > 0,
	\end{cases}
\end{equation}
utilizando la relación del resultado (\ref{eq:v-sub-p}) \cite{Leveque}. Entonces, la solución al problema de Riemann tiene la forma (\ref{eq:sol-linear-sys-con-caracteristicas}), siendo $v_p$ una función por partes.

Si se define a $P(x,t)$ como el valor máximo de $p$ tal que $x - \lambda_{p} t > 0$, entonces se puede escribir la solución al problema de Riemann como:
\begin{equation}
	\mathbf{U}(x,t) = \sum_{p=1}^{P(x,t)}\beta_{p}\mathbf{r}_{p} + \sum_{p=P(x,t) +1}^{m} \alpha_{p} \mathbf{r}_p
	\label{eq:sol-riemann-linear-sys-1}
\end{equation}
\cite{Leveque}.

Se puede verificar que esta solución cumple con la condición de salto de Rankine-Hugoniot. Notando que al atravesar una curva característica $p$, el coeficiente de $\mathbf{r}_p$ salta de acuerdo a la siguiente expresión:
\begin{equation}
	\mathbf{U}_{Lp} - \mathbf{U}_{Rp} = (\beta_{p} - \alpha_{p})\mathbf{r}_p
\end{equation}
donde $\mathbf{U}_{Lp}$ y $\mathbf{U}_{Rp}$ son los valores de la onda a la izquierda y derecha de la curva característica $p$, respectivamente \cite{Leveque}.

Tomando en cuenta que $\mathbf{F}(\mathbf{U}) = \mathbf{A}\mathbf{U}$, entonces:
\begin{equation}
 	\mathbf{F}(\mathbf{U}_{Lp}) - \mathbf{F}(\mathbf{U}_{Rp}) = \mathbf{A}(\mathbf{U}_{Lp} - \mathbf{U}_{Rp})
\end{equation}
\begin{equation}
	\mathbf{F}(\mathbf{U}_{Lp}) - \mathbf{F}(\mathbf{U}_{Rp}) = \mathbf{A}(\beta_{p} - \alpha_{p})\mathbf{r}_p
\end{equation}
\begin{equation}
	\mathbf{F}(\mathbf{U}_{Lp}) - \mathbf{F}(\mathbf{U}_{Rp}) = \lambda_{p}(\mathbf{U}_{Lp} - \mathbf{U}_{Rp})
\end{equation}
con lo cual se demuestra que se cumple la condición Rankine - Hugoniot, notando que la velocidad de propagación de la discontinuidad es $\lambda_p$ \cite{Leveque}.

Otra forma útil de escribir la solución (\ref{eq:sol-riemann-linear-sys-1}) es a través de los saltos previamente desarrollados, esto es,
\begin{equation}
	\mathbf{U}(x,t) = \mathbf{U}_{L}(x,t) + \sum_{\lambda_{p} < x/t} (\beta_{p} - \alpha_{p}) \mathbf{r}_{p} 
\end{equation}
o, de forma equivalente,
\begin{equation}
	\mathbf{U}(x,t) = \mathbf{U}_{R}(x,t) - \sum_{\lambda_{p} \geq x/t} (\beta_{p} - \alpha_{p}) \mathbf{r}_{p} 
\end{equation}
\cite{Leveque}.

\par}               % termina interlineado 1 1/2

\end{document}
