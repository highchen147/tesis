%%% INCLUYA SUS CONCLUSIONES Y RECOMENDACIONES


\chapter{CONCLUSIONES}
\begin{enumerate}
	\item El método de volúmenes finitos es óptimo para resolver ecuaciones diferenciales de conservación. Sin embargo, debe estudiarse adicionalmente un esquema apto para el cálculo de flujos numéricos aproximados entre las celdas definidas por el método.
	\item El esquema de Roe es un método práctico para resolver numéricamente las ecuaciones de Euler y a través de los promedios de Roe, su implementación es directa. La parte más difícil de completar para implementar el esquema de Roe en la resolución de un sistema de conservación, es encontrar el  promedio adecuado para los autovalores y autovectores de la matriz $\hat{\mathbf{A}}$. Sin embargo, las propiedades que se exigen que cumpla $\hat{\mathbf{A}}$ son sencillas de satisfacer y se pueden usar los promedios de Roe para las ecuaciones de Euler como base.
	\item Las diferencias entre las simulaciones de C++ y PyClaw son menores. Sin embargo, la solución de PyClaw se realiza utilizando técnicas de alta resolución; específicamente, produciendo soluciones a segundo orden de precisión. La principal desventaja al utilizar PyClaw es su ineficiente captura de ondas de rarefacción transónica, dado que no posee algoritmos de corrección de entropía implementados.
	\item El uso de las simulaciones de gases poliatómicos como experimentos, a través de la solución numérica de las ecuaciones de Euler, resulta ser una tarea que brinda intuición física suficiente para entender mejor los fenómenos de los medios continuos. A través de las simulaciones con distintos coeficientes de dilatación adiabática, se pudieron observar efectos que parecían alcanzables únicamente por medio de la solución analítica de los problemas. Además, las variables más abstractas, como la entropía, son más fáciles de comprender mediante una gráfica de la misma, obtenida a través de una simulación.
	\item Las simulaciones numéricas no reemplazan completamente los experimentos de laboratorio. Sin embargo, las primeras pueden ser de alta utilidad al hacer comparaciones de gases ficticios, tomando como ejemplo las simulaciones de gases con distintos $\gamma$. Las soluciones numéricas son un recurso de alta utilidad didáctica, especialmente en cursos básicos de la carrera de física, como termodinámica.
\end{enumerate}

\chapter{RECOMENDACIONES}
\begin{enumerate}
	\item Se recomienda estudiar a mayor profundidad el método de división de flujos, implementado en PyClaw, con la finalidad de añadir algún algoritmo que realice la corrección de entropía. De preferencia, agregar un conjunto de algoritmos correctores de entropía a disposición del solucionador de Euler. Esta adición es factible de realizar, ya que la librería Clawpack es un software de código abierto.
	\item Ya que se estudiaron brevemente algunos esquemas distintos al esquema de Roe, se sugiere adaptar el esquema de Godunov para resolver las ecuaciones de Euler. Esto puede realizarse en un programa de C++ o bien, añadiendo un solucionador exacto en el paquete de Clawpack. Sería apropiado analizar y comparar las soluciones producidos con los esquemas de Roe y Godunov.
	\item Es imprescindible considerar la implementación de métodos de impresión de datos más estandarizados, en el software de Clawpack. Esto se expone como una recomendación porque PyClaw resulta ser ligeramente limitante al momento de generar la solución numérica. Es importante reconocer, de todas maneras, que PyClaw se especializa en producir soluciones gráficas a través del paquete de visualización visClaw.
\end{enumerate}
