%%% INCLUYA LA SIMBOLOGÍA NECESARIA EN ESTE APARTADO
%%% NO CAMBIAR LA DEFINICIÓN DE LA TABLA LARGA


\chapter{LISTA DE SÍMBOLOS}

\begin{longtable}{@{}l@{\extracolsep{\fill}} p{4.65in} @{}}  %%%	NO CAMBIAR ESTA LÍNEA
  \textsf{Símbolo} & \textsf{Significado}\\[12pt]
  \endhead
  $F_{x}$ & derivada parcial de $F$ respecto a $x$\\
  $\mathbf{U}$ & vector de magnitudes conservadas \\
  ${U}$ & vector de magnitudes conservadas aproximadas numéricamente \\
  $\mathbf{F}$ & vector de flujos de magnitudes conservadas exactas \\
  ${F}$ & vector de flujos de magnitudes conservadas aproximados numéricamente \\
  $\mathbf{A}$ & matriz jacobiana \\
%  $\alpha$ & velocidad de advección \\
  $\varepsilon$ & coeficiente de viscosidad cinemática\\
  $u$ & velocidad del gas sobre el eje $x$\\
  $\rho$ & densidad del gas\\
  $p$ & presión del gas\\
  $e$ & energía interna específica del gas\\
  $E$ & densidad de energía total del gas\\ 
  $T$ & temperatura del gas \\
  $W$ & trabajo realizado sobre el gas \\
  $k_B$ & constante de Boltzmann \\
  $\mathcal{R}$ & constante específica de los gases\\
  $c_v$ & capacidad calorífica específica a volumen constante\\
  $c_p$ & capacidad calorífica específica a presión constante\\
  $\gamma$ & coeficiente de dilatación adiabática\\
  $\alpha$ & grados de libertad internos del gas\\
  $S$ & entropía del gas
\end{longtable}
