\chapter{ECUACIONES DE EULER Y APLICACIÓN DEL ESQUEMA DE ROE}

En este capítulo se explican y derivan las ecuaciones de Euler utilizando las variables generales (presión, densidad y velocidad) y se introducen las variables conservadas. Se explican las ligaduras adicionales involucradas para que las ecuaciones de Euler sean aplicadas a un gas ideal poliatómico.

Se describe el esquema de Roe implementado en la solución de las ecuaciones de Euler para un gas ideal poliatómico así como las demás especificaciones requeridas por el método de volúmenes finitos. Se explica la implementación del método numérico en \texttt{C++}. Se muestran los resultados obtenidos para un problema de condiciones iniciales específicas.
\section{Ecuaciones de Euler}
Las ecuaciones fundamentales de la dinámica de fluidos se basan en las siguientes leyes de conservación universales:
\begin{itemize}
	\item Conservación de la masa
	\item Conservación del momentum
	\item Conservación de la energía.
\end{itemize}
La ecuación de conservación de la masa, que se derivó en la sección (\ref{sec:derivacion-continuidad}), consiste en aplicar la ecuación de continuidad para un fluido con cierta densidad. La ley de conservación del momentum resulta al aplicar la Segunda Ley de Newton en un fluido. Por último, la ley de conservación de la energía es equivalente a la aplicación de la Primera Ley de la Termodinámica. Además de las tres ecuaciones de conservación, es necesario establecer una relación entre las variables físicas del fluido, de tal manera que el sistema de ecuaciones sea resoluble \cite{heattransfer}. En el caso de un gas ideal, la ecuación adicional que relaciona las variables densidad $\rho$, presión $p$ y temperatura $T$, es la ecuación de estado. 
\subsection{Derivación de las ecuaciones}
%\subsubsection{Ecuación de continuidad}
La ecuación de continuidad, derivada en la sección (\ref{sec:derivacion-continuidad}) tiene la siguiente forma
\begin{equation}
	\rho_{t} + (\rho v)_{x} = 0.
	\label{eq:continuidad-euler}
\end{equation}
Ya que el flujo asociado a esta ecuación de conservación es $\rho v$, se puede interpretar que, generalmente, para cualquier cantidad física $z$ que sufra advección, su flujo estará dado por el producto de la cantidad por la velocidad de advección, i.e., $f=zv$. 

A partir del último razonamiento, el flujo asociado a la advección de momentum $\rho v$, tiene una contribución al flujo dada por $(\rho v) \cdot v = \rho v^2$. Sin embargo, además de la advección que sufre el momentum, deben considerarse las fuerzas que actúan en el fluido para expresar el flujo total del momentum. En este texto no se considerarán fuerzas externas, por lo que únicamente habría que tomar en cuenta la fuerza interna del fluido, que está dada por el gradiente de \textbf{presión}, $p_x$ \cite{LeVeque}. De esta manera, se consigue la ecuación de conservación del momentum:
\begin{equation}
	(\rho v)_t + (\rho v^{2} + p)_{x} = 0.
	\label{eq:momentum-euler}
\end{equation}

Para derivar la conservación de energía, se debe considerar que ésta se compone por un término cinético y uno correspondiente a la energía interna del fluido. Sea $E$ la densidad de energía total de un fluido. Entonces, se tiene que:
\begin{equation}
	E = \frac{1}{2} v^{2} + e,
	\label{eq:energia}
\end{equation}
donde el término $\frac{1}{2} v^{2}$ corresponde a la densidad de energía cinética por unidad de masa y $e$ es la \textbf{energía interna} por unidad de masa del fluido, que también suele denominarse como energía interna específica. La energía interna depende de los grados de libertad internos de las moléculas de los gases, considerando energía de rotación, cinética, de vibración y otras formas más complejos de energía. Las ecuaciones de Euler asumen que el fluido se encuentra en equilibrio termodinámico y que la ecuación de estado provee una expresión para la energía interna específica tal que ésta depende de la presión y la densidad del fluido únicamente
\begin{equation}
	e = e(p, \rho).
\end{equation}
De igual manera que con el momentum, la energía total se ve afectada por la advección del flujo del fluido. Dado que no se consideran fuerzas externas que afecten al sistema, únicamente la presión del fluido hace trabajo y es proporcional al gradiente de $vp$. Entonces la ecuación de conservación de la energía se reduce a:
\begin{equation}
	(\rho E)_{t} + [v(\rho E + p)]_{x} = 0.
	\label{eq:energia-euler}
\end{equation}
haciendo notar que $\rho E$ corresponde a la energía total del gas.

Las ecuaciones de Euler, (\ref{eq:continuidad-euler}), (\ref{eq:momentum-euler}) y (\ref{eq:energia-euler}) se pueden escribir en forma vectorial, obteniendo:
\begin{equation}
	\begin{bmatrix}
		\rho \\
		\rho v \\
		\rho E
	\end{bmatrix}_{t}
+
	\begin{bmatrix}
		\rho v \\
		\rho v^{2}+p \\
		v(\rho E + p)
	\end{bmatrix}_{x}
= 0,
\label{eq:euler-eqs-1}
\end{equation}
cuya forma coincide con la presentada en (\ref{eq:conservacion}). Cabe mencionar que en este texto se considerarán problemas unidimensionales solamente.

%Para escribir el sistema (\ref{eq:euler-eqs-1}) de la forma (\ref{eq:conservacion-jacobiana}) 

\subsection{Ecuación de estado para un gas politrópico}
Para completar el sistema de ecuaciones de Euler, es necesario definir la ecuación de estado que relacione la energía con las variables físicas de presión y  densidad. El desarrollo de la ecuación de estado para un gas ideal sigue de cerca la sección \textit{Ideal Gas} del capítulo \textit{Some Linear Systems} en \cite{LeVeque}.

La energía interna de un gas idea es únicamente dependiente de la temperatura,
\begin{equation}
	e = e(T).
\end{equation}
Mientras que la temperatura se relaciona con la presión y densidad a través de la ley del gas ideal, que define al mismo. Esta es:
\begin{equation}
	p = \mathcal{R}\rho T
	\label{eq:ideal-gas-law}
\end{equation}
donde $\mathcal{R}$ es la constante específica de los gases, que se define como el cociente entre la constante de Boltzmann $k_B$ y  la masa de cada molécula del gas $m$ \cite{blundell}. Por otro lado, la energía interna específica es proporcional a la temperatura,
\begin{equation}
	e = c_{v}T
	\label{eq:energia-interna-1}
\end{equation}
donde $c_{v}$ es la capacidad calorífica específica a volumen constante. Los gases que cumplen con esta propiedad se conocen como gases \textbf{politrópicos}. Entonces, si se cambiase la temperatura de un gas en una cantidad infinitesimal $\dd{T}$, manteniendo el volumen constante, se obtendría
\begin{equation}
	\dd{e} = c_{v} \dd{T}
\end{equation}
como cambio infinitesimal de energía interna específica. En cambio, si se permite que el gas se expanda pero manteniendo ahora la presión constante, se obtendría una expresión para el cambio de energía considerando al trabajo, $W = -p\dd{V}$, realizado sobre el gas, esto es
\begin{equation}
	m\dd{e} = \dd{W} + \dbar Q
\end{equation}
\begin{equation}
	m\dd{e} = -p\dd{V} + \dbar Q
\end{equation}
\begin{equation}
	\dd{e} = -\frac{p}{m}\dd{V} + c_{p}\dd{T}
\end{equation}
y puesto que al usar $\rho = \frac{m}{V}$, se obtiene que $\dd{V} = m\dd(\frac{1}{\rho})$, entonces:
\begin{equation}
	\dd{e} = -p\dd(\frac{1}{\rho}) + c_{p}\dd{T}
\end{equation}
\begin{equation}
	\dd(e + \frac{p}{\rho}) = c_{p}\dd{T},
	\label{eq:entalpia-temp-dif}
\end{equation}
donde $c_p$ es la capacidad calorífica específica del gas a presión constante. A partir del anterior resultado se define la \textbf{entalpía} interna $\mathrm{h}_i$:
\begin{equation}
	\mathrm{h}_i \equiv e + \frac{p}{\rho},
	\label{eq:entalpia-1}
\end{equation}
mientras que la entalpía total $\mathrm{h}$ \footnote{No confundir con $h$ definida en \ref{eq:def-h} como el tamaño de cada celda del dominio.} es:
\begin{equation}
	\mathrm{h} = E + \frac{p}{\rho}.
	\label{eq:entalpia-total}
\end{equation}
Para un gas politrópico se considera a $c_p$ como constante, por lo que integrando la ecuación (\ref{eq:entalpia-temp-dif}) se obtiene otra expresión para la entalpía interna,
\begin{equation}
	\mathrm{h}_i = c_p T.
	\label{eq:entalpia-2}
\end{equation}
Por otro lado, de acuerdo a la ley del gas ideal se tiene la siguiente relación
\begin{equation}
	c_p - c_v = \mathcal{R}.
	\label{eq:diferencia-capacidades}
\end{equation}
Para continuar con la derivación, es conveniente definir el \textbf{coeficiente de dilatación adiabática} $\gamma$ como:
\begin{equation}
	\gamma \equiv \frac{c_p}{c_v}.
	\label{eq:gamma-1}
\end{equation}
Dicha cantidad está estrechamente relacionada con el número de grados de libertad internos del gas, que depende de la naturaleza del mismo. Según el teorema de equipartición de la energía, el promedio de energía involucrada en cada grado de libertad es el mismo. Específicamente, cada grado de libertad aporta una cantidad promedio de energía por molécula de $\frac{1}{2}k_B T$. De tal manera que si existen ${\alpha}$ grados de libertad internos en un gas que tiene $n$ moléculas por unidad de masa, se obtiene una expresión para la energía interna específica
\begin{equation}
	e = \frac{\alpha}{2}n k_B T
\end{equation}
o bien, 
\begin{equation}
	e = \frac{\alpha}{2}\mathcal{R} T.
	\label{eq:energia-interna-2}
\end{equation}
Comparando con (\ref{eq:energia-interna-1}) se obtiene
\begin{equation}
	c_v = \frac{\alpha}{2}\mathcal{R},
\end{equation}
sustituyendo en (\ref{eq:diferencia-capacidades}),
\begin{equation}
	c_p = \left(1+\frac{\alpha}{2}\right)\mathcal{R}.
\end{equation}
Y al aplicar la definición de $\gamma$ en (\ref{eq:gamma-1}) se obtiene la expresión
\begin{equation}
	\gamma = \frac{\alpha + 2}{\alpha},
\end{equation}
que como se expuso previamente, relaciona el número de grados de libertad con el coeficiente de dilatación adiabática. Por ejemplo, para gases monoatómicos se consideran únicamente tres grados de libertad, correspondientes al movimiento traslacional en tres dimensiones, por lo que $\alpha = 3$ y $\gamma = 5/3$. En gases diatómicos (como el aire, compuesto por $H_2$ y $N_2$ principalmente) se agregan dos grados libertad correspondientes a dos ejes de rotación posibles, de tal manera que $\alpha = 5$ y $\gamma = 7/5 = 1.4$.

Por último, se escribe la ecuación de estado del gas ideal para la energía utilizando  (\ref{eq:ideal-gas-law}),
\begin{equation}
	e = c_v T = \frac{c_v}{\mathcal{R}}\cdot\frac{p}{\rho}
\end{equation}
\begin{equation}
	e = \frac{p}{(\gamma - 1) \rho}.
\end{equation}
De tal manera que la energía total del gas ($\rho E$) queda como
\begin{equation}
	 \rho E = \frac{1}{2}\rho v^{2} + \frac{p}{\gamma - 1}.
\end{equation}
\section{Aplicación del esquema de Roe}
\subsection{Variables conservadas y propiedades de $\mathbf{A(\mathbf{U})}$}
Para implementar el método de volúmenes finitos junto al esquema de Roe en la solución numérica de las ecuaciones de Euler es necesario identificar las \textbf{variables conservadas} involucradas en la definición de un sistema general de conservación, definido en (\ref{eq:conserv-deriv-short}). Comparando esta expresión con el sistema (\ref{eq:euler-eqs-1}) se obtiene:
\begin{equation}
	\mathbf{U} = 
	\begin{bmatrix}
		\mathbf{u}_1 \\
		\mathbf{u}_2 \\
		\mathbf{u}_3
	\end{bmatrix} \equiv
	\begin{bmatrix}
		\rho \\
		\rho v \\
		\rho E
	\end{bmatrix}
\end{equation}
\begin{equation}
	\mathbf{F} = 
	\begin{bmatrix}
		\mathbf{f}_1 \\
		\mathbf{f}_2 \\
		\mathbf{f}_3
	\end{bmatrix} \equiv
	\begin{bmatrix}
		\rho v \\
		\rho v^2 + p \\
		v(\rho E + p)
	\end{bmatrix}.
\end{equation}
Por tanto, se define a $\mathbf{u}_i$ como la i-ésima variable conservada cuyo  flujo correspondiente, $\mathbf{f}_i$. 

Por otro lado, el esquema de Roe depende de los autovalores y autovectores de la matriz jacobiana $\mathbf{A(\mathbf{U})}$ definida en (\ref{eq:conservacion-jacobiana}). Aplicando la definición (\ref{eq:jacobiana-A-definicion}) mientras se toma en cuenta la ecuación de estado para la energía (\ref{eq:ideal-gas-law}), se obtiene una expresión para esta matriz:
\begin{equation}
	\mathbf{A(\mathbf{U})} =
	\begin{bmatrix}
		0 & 1 & 0 \\
		\frac{1}{2}(\gamma - 3)v^2 & (3  - \gamma) v & (\gamma - 1) \\
		\frac{1}{2}(\gamma - 1)v^3 - v(\rho E+p)/\rho & (\rho E+p)/\rho-(\gamma - 1)v^2 & \gamma v \\
	\end{bmatrix}.
\end{equation}
Entonces, se deben calcular los autovalores y autovectores de esta matriz. Siguiendo la notación de los últimos capítulos, al autovalor $\lambda_i$ le corresponde el autovector $\mathbf{r}_i$, y estos son:
\begin{equation}
	\lambda_1 = v-c, \hspace{4mm}
	\lambda_2 = c, \hspace{4mm}
	\lambda_3 = v+c,
\end{equation}
\begin{equation}
	\mathbf{r}_1 = 
	\begin{bmatrix}
		1 \\
		v-c \\
		\mathrm{h}-vc
	\end{bmatrix},\hspace{4mm}
	\mathbf{r}_2 = 
	\begin{bmatrix}
		1 \\
		v \\
		\frac{1}{2}v^{2}
	\end{bmatrix},\hspace{4mm}
	\mathbf{r}_3 = 
	\begin{bmatrix}
		1 \\
		v+c \\
		\mathrm{h}+vc
	\end{bmatrix}.
\end{equation}
donde $c$ es la velocidad del sonido, dada por:
\begin{equation}
	c = \sqrt{\frac{\gamma p}{\rho}},
\end{equation}
mientras $\mathrm{h}$ es la entalpía total definida en (\ref{eq:entalpia-total}), o bien, equivalentemente:
\begin{equation}
	\mathrm{h} = \frac{1}{2}v^2 + \left(\frac{\gamma}{\gamma - 1}\right)\frac{p}{\rho}.
\end{equation}
Por lo tanto, es posible escribir la velocidad del sonido en términos de la entalpía y la velocidad:
\begin{equation}
	c^2 = (\gamma - 1)\left[\mathrm{h} - \frac{1}{2}v^{2}\right].
\end{equation}
\subsection{Valores promediados de Roe}
Retornando a la expresión para el flujo numérico simple de Roe,
\begin{equation}
	F(\mathbf{U}_L, \mathbf{U}_R) = \frac{1}{2}\left(\mathbf{F}(\mathbf{U}_L) +\mathbf{F}(\mathbf{U}_R)\right) - 
	\frac{1}{2}\sum_{p=1}^{m}|\hat{\lambda}_{p}|\alpha_{p}\mathbf{\hat{r}}_{p},
	\label{eq:roe-flux-2}
\end{equation}
se destaca que es necesario encontrar los autovalores $\hat{\lambda}_i$ y autovectores $\hat{r}_i$ correspondientes a la matriz aproximada $\mathbf{\hat{A}}$. Roe \cite{roe81} propuso expresiones, basadas en promedios específicos de los valores adyacentes, para los mencionados autovectores y autovalores de la matriz aproximada y junto a Pike demostraron que son los únicos promedios que satisfacen con las condiciones necesarias impuestas al flujo numérico del esquema de Roe \cite{roe86}. Los autovalores y autovectores de Roe son:
\begin{equation}
	\hat{\lambda}_1 = \tilde{v}-\tilde{c}, \hspace{4mm}
	\hat{\lambda}_2 = \tilde{c}, \hspace{4mm}
	\hat{\lambda}_3 = \tilde{v}+\tilde{c},
\end{equation}
\begin{equation}
	\mathbf{r}_1 = 
	\begin{bmatrix}
		1 \\
		\tilde{v}-\tilde{c} \\
		\tilde{\mathrm{h}}-\tilde{v}\tilde{c}
	\end{bmatrix},\hspace{4mm}
	\mathbf{r}_2 = 
	\begin{bmatrix}
		1 \\
		\tilde{v} \\
		\frac{1}{2}\tilde{v}^{2}
	\end{bmatrix},\hspace{4mm}
	\mathbf{r}_3 = 
	\begin{bmatrix}
		1 \\
		\tilde{v}+\tilde{c} \\
		\tilde{\mathrm{h}}+\tilde{v}\tilde{c}
	\end{bmatrix}.
\end{equation}
 Mientras que los coeficientes de las características $\alpha_{i}$ están dados por:
 \begin{equation}
 	\alpha_1 = \frac{1}{2\tilde{c}^{2}}[\Delta p - \tilde{\rho}\tilde{c}\Delta v], \hspace{4mm}
 	\alpha_2 = \frac{1}{\tilde{c}^{2}}[\tilde{c}^{2}\Delta\rho - \Delta p], \hspace{4mm}
 	\alpha_1 = \frac{1}{2\tilde{c}^{2}}[\Delta p + \tilde{\rho}\tilde{c}\Delta v],
 \end{equation}
con
\begin{equation}
	\tilde{\rho} = \sqrt{\rho_L \rho_R}
\end{equation}
\begin{equation}
	\tilde{v} = \frac{\sqrt{\rho_{L}}\cdot v_L + \sqrt{\rho_{R}}\cdot v_R}{\sqrt{\rho_{L}} + \sqrt{\rho_{R}}}
\end{equation}
\begin{equation}
	\tilde{\mathrm{h}} = \frac{\sqrt{\rho_{L}}\cdot \mathrm{h}_L + \sqrt{\rho_{R}}\cdot \mathrm{h}_R}{\sqrt{\rho_{L}} + \sqrt{\rho_{R}}}
\end{equation}
\begin{equation}
	\tilde{c}^{2} = (\gamma - 1)[\tilde{h} - \tfrac{1}{2}\tilde{v}^2]
\end{equation}
donde $\Delta u = u_R - u_L$, para cualquier variable $u$, siendo $u_R$ y $u_L$ el valor de la variable a la derecha e izquierda de la interfaz en donde se calcula el flujo, respectivamente. 
\subsection{Código implementado}
\section{Resultados}
