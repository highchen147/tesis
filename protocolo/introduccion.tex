%%% Haga el diseño que más le guste
\chapter{INTRODUCCIÓN}
El estudio de las ecuaciones diferenciales es de gran importancia en las ciencias físicas, ya que cada teoría física se sustenta en ecuaciones diferenciales que describen el comportamiento a través del tiempo de cualquier sistema que dicha teoría busque explicar. La motivación del estudio de las ecuaciones diferenciales es encontrar soluciones generales de las mismas, principalmente a través de métodos analíticos que buscan soluciones exactas de las ecuaciones diferenciales. Sin embargo, no todas las ecuaciones diferenciales poseen soluciones exactas, lo cual motiva el estudio y desarrollo de métodos numéricos para la resolución de las mismas.

En el área de estudio del análisis numérico aplicado a ecuaciones diferenciales, existe una gran variedad de métodos y esquemas que se aplican para obtener una solución numérica, esto se debe a la amplia variedad de ecuaciones diferenciales de la física que carecen de solución analítica. Por otro lado, las ecuaciones diferenciales parciales son considerablemente más complejas que las ecuaciones diferenciales ordinarias, por lo que existen métodos más apropiados para resolver ecuaciones diferenciales que involucran funciones de varias variables.

Las ecuaciones de conservación tienen un papel importante en múltiples áreas de la física, de tal manera que se han desarrollado métodos numéricos apropiados para resolver este tipo de ecuaciones diferenciales parciales, siendo el método de volúmenes finitos el más utilizado. Un conjunto en particular de ecuaciones de conservación son las ecuaciones de Euler, que rigen la dinámica de un fluido compresible y no viscoso a partir de su ecuación de estado. Existen pocas soluciones analíticas conocidas a las ecuaciones de Euler, por lo que resolver este conjunto de ecuaciones de conservación con un método numérico apropiado resulta ser un problema interesante.