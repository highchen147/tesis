%%% Haga el diseño que más le guste
\chapter{INTRODUCCIÓN}
El estudio de las ecuaciones diferenciales es de gran importancia en las ciencias físicas, ya que cada teoría física se sustenta en ecuaciones diferenciales que describen el comportamiento a través del tiempo de cualquier sistema que dicha teoría busque explicar. La motivación del estudio de las ecuaciones diferenciales es encontrar soluciones generales de las mismas, principalmente a través de métodos analíticos que buscan soluciones exactas de las ecuaciones diferenciales. Sin embargo no todas las ecuaciones diferenciales poseen soluciones exactas, lo cual motiva el desarrollo de métodos numéricos para la resolución de las mismas.

En el área de estudio del análisis numérico, aplicado a ecuaciones diferenciales, existe una gran variedad de métodos y esquemas de solución, dada la amplia variedad de ecuaciones diferenciales que se hallan en la física. Además, las ecuaciones diferenciales parciales son considerablemente más difíciles de resolver que las ecuaciones diferenciales ordinarias, por lo que existen otros métodos más apropiados para resolver ecuaciones diferenciales que involucran funciones de varias variables.

%. Esto ha conducido al estudio de múltiples métodos analíticos que tienen como objetivo encontrar las soluciones exactas de estas ecuaciones diferenciales. Sin embargo, no todas las ecuaciones diferenciales tienen soluciones analíticas, lo que conlleva a la necesidad de desarrollar métodos para encontrar soluciones numéricas. Es así como nace la rama del análisis numérico aplicado a la resolución de ecuaciones diferenciales.