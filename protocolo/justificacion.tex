\chapter{JUSTIFICACIÓN}
Este trabajo de investigación busca determinar y estudiar la solución de las ecuaciones de Euler en forma numérica. Como se mencionó previamente, las ecuaciones de Euler tienen pocas soluciones analíticas conocidas y por tanto la solución numérica de estas ecuaciones es una forma alternativa de realizar experimentos con gases ideales. Por esta razón, se incluyó también como objetivo específico la aplicación de la solución numérica con el fin de encontrar las diferencias principales en la evolución temporal de gases con diferentes coeficientes de dilatación adiabática $\gamma$, que además, es un valor que está relacionado con los grados de libertad internos del mismo gas ideal.

Se consideró, como objetivo específico, la comparación de la solución obtenida con el programa escrito en \texttt{C++} con la solución producida a través de la librería PyClaw. Esta comparación tiene la intención de corroborar el funcionamiento adecuado del programa y determinar algunas ventajas y desventajas de la implementación numérica en el lenguaje \texttt{C++}.

Por último, cabe destacar que la aplicación del esquema de Roe y el método de volúmenes finitos en las ecuaciones de Euler es un tema importante de estudio en el área del análisis numérico ya que no solo puede aplicarse en el ya mencionado sistema de ecuaciones diferenciales, sino que se puede implementar en cualquier sistema de ecuaciones de conservación de la física y lograr modelar numéricamente otros sistemas complejos de otras áreas de la física.